Sendo assim, no nosso modelo a opinião do agente vai ser dada por uma função de
distribuição de probabilidade:

\[ f(\theta) = \frac{1}{\sqrt{2 \pi} \sigma_i} e^{-
    \frac{(\theta - x )^2}{2 \sigma_i}} \]






\section{Definição da variável U}

Introduzimos uma probabilidade p que o outro agente saiba algo sobre $\theta$ isto é,
que tenha uma opinião plausível e que vale a pena ser levada em conta ; e vai
ter uma chance 1 - p que o outro agente não tenha informação sobre $\theta$ de forma
que o \textit{likelihood} de $j$ estar correto é dado por $ f(x_j|\theta) = p
N(\theta,\sigma_j^2) + (1-p)U(0,1)$.

Temos então que a distribuição da nova opinião é dada por uma mistura
de duas normais com médias diferentes, que é resultante da
multiplicação do likelihood e o prior normal de $i$:
  
  \begin{align*}
    f(\theta | x_j)
    \propto 
    p
    e^
    {-(\frac{1}{2\sigma_i^2})
    [(\theta - x_i)^2
    +
    (x_j - \theta )^2
    ]}
    +
    (1-p)
    e^{-\frac{(x_i - x_j)^2}{(2 \sigma_i^2)}}
  \end{align*}


   Se calcularmos o $E(x_i)$ da expressão anterior , a nova opinião,
  temos:
  \begin{align*}
    x_i(t+1)
    =
    p
    \frac{x_i(t) + x_j(t)}{2}
    +
    (1-p^*)x_i(t)
  \end{align*}
  Caso queiramos atualizar a incerteza também temos que usar a
  seguinte equação:
  \begin{align*}
    \sigma_i^2(t+1)
    =
    \sigma_i^2(t)
    (1 - \frac{p^*}{2})
    +
    p^*
    (1-p^*)
    (\frac{x_i(t)-x_j(t)}{2})^2
  \end{align*}

  Onde $p^*$ é :

  \begin{align*}
    p^*
    =
    \frac{
      p \frac{1}{\sqrt{2 \pi} \sigma_i}
      e^{(- \frac{x_i (t) - x_j (t))^2}{2 \sigma_i^2}}
    }{
      p
      \frac{1}{\sqrt{2 \pi} \sigma_i}
      e^{(- \frac{x_i (t) - x_j (t))^2}{2 \sigma_i^2}}
    }
  \end{align*}

  \section{Estrutura do Modelo}

  Sendo assim temos a seguinte estrutura:

  \begin{itemize}
  \item Uma população de N \textbf{A}gentes;
  \item Uma \textbf{T}opologia de formato grade regular com
    vizinhanças $V_n = 4, 6 \text{ ou } 8 $;
  \item \textbf{I}nteração par a par : a cada passo de tempo os
    agentes sorteiam um de seus vizinhos e interagem\footnote{Uma
      alternativa é que a cada passo de tempo só dois agentes
      interagem e são vizinhos.};
  \item \textbf{O}pinião derivada de um \textit{prior} normal , onde o ponto
    ideal/opinião declarada é a média e a incerteza é o desvio padrão;
  \item Regra de Atualização (\textit{\textbf{U}pdating Rule}):
    Atualiza-se sua opinião por meio da multiplicação do \textit{prior}
    multiplicação com um \textit{likelihood} que leva em conta a
    possibilidade de que o vizinho não influencie o agente;
  \end{itemize}




  
\section{Modelos em Política}

Fica claro então que a área de dinâmicas de opinião é ampla e volumosa, mas qual
sua relação com a política? Temos aqui de diferenciar o sistema alvo, que
buscamos representar e compreender, e as ferramentas para analisar o sistema
alvo, os modelos.


A Ciência Política faz uso de um conjunto de modelos utilizados para representar
fenômenos que chamamos de políticos, o sistema alvo. Os cientistas usam modelos
teóricos com o objetivo de ter \textit{insights}, por meio da analogia, sobre os
processos geradores de dados subjacentes aos fenômenos
\cite{clarke2012model,morton1999methods}. Sendo assim modelos de OD podem
interagir tanto com os modelos já usados, ou serem aplicados diretamente ao
sistema alvo, sem a mediação da teoria tradicional. Isso permite uma grande gama
de abordagens na interface entre as áreas, ainda mais quando política, o objeto,
não é monopólio da Ciência Política, sendo estudada por diversas disciplinas,
como Sociologia, Economia, Antropologia, Psicologia, dentre outras.


Vamos discutir 3 trabalhos em OD, por buscarem dialogar com a literatura em Ciência
Política. O modelo de \citeonline{gomes2014}; o modelo de
\citeonline{pulick2016} ; e \citeonline{lorenz2017modeling}.


O modelo de \citeonline{gomes2014} é uma modificação do modelo de Axelrod. Nele
há 15 features, com traits binários. As vizinhanças definidas são neumann, moore
e grafo completo (global). Além disso existem dois parâmetros: ativa-confiança -
indica se a exposição a opiniões diferentes da do agente vão influencia-lo ;
ativa-axelrod - delimita um limiar para que haja interação (ativa-homofilia
deveria ser...). A regra de interação é entre pares $ij$ (como voter e axelrod)
e seguindo axelrod a interaçao ocorre se o limiar for atingido (isso se o
parâmetro ativa axelrod estiver ligado, senão a interação necessariamente vai
ocorrer). A regra de atualização é acrescentar +1 ou -1 caso a opinião do agente
$i$ se aproxime da do agente $j$. Se houver confiança o update depende da
proporção de opiniões similares, se 30\% das opiniões sao iguais então aumenta
em 0.3 ao invés de 1. A confiança não determina se há ou não interação, mas
influi no quanto de atualização ocorre.

O modelo de \citeonline{pulick2016} também é uma modificação no modelo de
Axelrod. Os agentes tem uma opinião multivariada e contínua - é uma lista de 5
números opinion topics, que podem variar de 1 a 9, os ``beliefs''. Existem três
tipos de interação - interação entre vizinhos ( pares $ij$, mas a vizinhança é
moore ) , interação com a mídia ( que no caso é um agente externo extremista) ,
e interações de grupo (algo a la majority model = uma posição aleatória é
escolhida, um `$r$ é escolhido, e agentes dentro desse $r$ fazem parte do
meeting. Tira-se uma média do grupo e cada agente ajusta suas crenças para mais
a metade da crença do grupo.). A interação só ocorre se for atingido um limiar
de homofilia entre o agente e o alvo (para vizinhos e mídia). A similaridade é
calculada pela soma das diferenças entre os beliefs normalizado pela diferença
máxima possível e subtraído de 1. Se a similaridade entre o agente e o alvo é x
ele tem x\% de chance de mudar um de seus beliefs.

Uma diferença fundamental para o modelo de Axelrod é que nesse modelo um feature
com trait value de 8 não é mais próximo de 9 do que de 2. Já no modelo de
\citeonline{pulick2016} ,uma série 8.027.563.190.615.97 é uma configuração de
posições em espaços contínuos. Isso quer dizer que nesse caso 8 é sim mais
próximo de 9 do que de 2, de forma que agentes interagem com uma probabilidade
derivada de sua similaridade, mas essa similaridade é mensurada não em termos de
pareamento exato em algum dos features, mas na proximidade do conjunto inteiro.
No modelo de Axelrod quando os agentes não tem trait match em algum feature a
probabilidade de interação é zero, enquanto que no modelo de
\citeonline{pulick2016} a similaridade é mensurada como distância de crença,
nunca sendo completamente zero. Ademais eles modificam a regra de atualização
também, ao invés da opinião mudar para à do alvo, é acrescido metade da
distancia da opinião entre o $a_i$ e $a_j$.


Por fim temos o modelo de \citeonline{lorenz2017modeling}. Nesse artigo Lorenz
propõe um modelo que gere a distribuição de preferências de eleitores. Ele tem
por ``background'' a literatura em ciência política sobre competição política
espacial, na qual partidos competem sobre um espaço delimitado pela preferência
dos eleitores. Essa literatura tende a assumir que para cada dimensão os
eleitores tem uma distribuição normal de preferências, estática. Contudo, Lorenz
demonstra, usando dados do ESS, que na verdade a distribuição tende a
ter\todo[color = yellow!10]{Vou ajeitar esse paragrafo!}
múltiplos picos, e é dificilmente aproximável por uma função a la a gaussiana ,
ou até mesmo a distribuição beta. Além disso essa distribuição muda ao longo do
tempo, mesmo que lentamente. Essas limitações da literatura serão esmiuçadas
mais à frente.

Sendo assim ele propõe-se a apresentar um modelo de dinâmicas de opinião que
gere uma distribuição unidimensional de preferências dos eleitores. O modelo que
ele propõe é um modelo de Deffuant modificado. Nele, uma população N de agentes
$a_i$ têm posição ideológica modelada como uma variável $x_i \in [0,1]$. Cada
agente também tem um limiar de confiança $\epsilon \in [0,1]$, homogêneo na população e
imutável, o qual determina o máximo de distância ideológica em relação a outro
agente $a_j$ que $a_i$ aceita quando interage com ele. Sendo assim as posições
ideológicas da população num tempo $t$ forma um vetor $x(t) \in [0,1]^N$. O $\epsilon$ é
uma forma de modelar homofilia. A regra de interação é entre pares $ij$ ( a la
voter) um agente $a_i$ é escolhido aleatoriamente e é pareado com um outro
agente escolhido aleatoriamente $a_j$. São duas regras de atualização: se os
agentes interagem, o agente $a_i$ assume a média da opinião dele com $a_j$. Além
disso Lorenz adiciona ruído, de forma que a cada ``tick'' os agentes
reconsideram sua opinião segundo uma probabilidade $p$, pequena (0.1) e retirada
de uma distribuição uniforme.
