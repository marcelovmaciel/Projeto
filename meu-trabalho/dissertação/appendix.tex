O Capítulo 1 suscita algumas definições formais apresentadas a seguir.

\section*{Propriedades Lógicas da Preferência e Otimização de Utilidade}

 As seguintes propriedades definem a
noção lógica de relação de preferência \cite{sep-preferences}:

\begin{enumerate}
\item \textit{Assimetria da preferência}: \( x \succ y \to \neg (y \succ x )\); 
\item \textit{Simetria de indiferença}: \(x \sim y \to  y \sim x\); 
\item \textit{Reflexividade da indiferença }: \(x \sim x\); 
\item \textit{Incompatibilidade entre preferência e indiferença}: \(x \succ y \to \neg ( x
  \sim y)\).
\end{enumerate}


No modelo do ator racional o agente age ``como se '' estivesse maximizando sua
função de utilidade  pelo fato da alternativa
preferida, ou ótima, por ele ser dada por \cite{binmore2008rational}:
\[\max_{\substack{x \in  X}}
  u_i(x)
\]

\section*{Regra de Bayes e Utilidade Esperada}

Suponha que um agente quer atualizar sua crença
sobre uma alternativa \(x \in X\), tendo em vista a observação de um dado \(m\).
Se ele atualizar sua crença segundo o Teorema de Bayes temos
que:

\begin{align*}
  p(x|m) =
  \frac{p(m|x)
  p(x)}
  {\int p(m|x)
  p(x)
  dx}
\end{align*}


Um corolário de agentes que têm preferências e crenças consistentes é que vão
agir segundo o \textit{princípio da utilidade esperada}
\cite{sep-rationality-normative-utility}. Para tal pressupõe-se que agentes vão
ter uma relação de preferência sobre \textit{apostas} \cite{jehle2001advanced},
onde o conjunto de apostas \(\mathcal{G}\) em \(X = \{ x_1, \ldots, x_n \}\) é dado
por:

\begin{align*}
  \mathcal{G} \equiv \big{\{}
  (p_1 \circ x_1, \ldots, p_n \circ x_n  )
  | p_i \geq 0,
  \sum_{i = 1 }^n p_i
  = 1  \big{\}}  
\end{align*}

Sendo assim, quando as alternativas são probabilísticas a utilidadde \(u:
\mathcal{G} \to \mathbb{R} \) do agente é
\cite{jehle2001advanced,sep-rationality-normative-utility}:

\begin{align*}
  u(\mathcal{G}) = \sum_{i =1}^n p_i u(x_i)
\end{align*}