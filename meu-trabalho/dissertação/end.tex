O trabalho teve por objetivo estudar a geração de distribuições de
posicionamento ideológicos em uma sociedade artificial, por meio da técnica da
modelagem baseada em agentes. Com isso buscou conectar o \textit{background}
conceitual usual da Ciência Política no tocante à modelagem formal de política
com a literatura em Dinâmicas de Opinião. Esse exercício ilustrou a importância
de pensar o posicionamento político como resultante de acomodações em diversas
questões. Particularmente, pensar o posicionamento político como composto de
opiniões em várias questões apontou para o papel do número de questões para
tornar mais robusto o efeito de interações assimilativas, enviesadas, ao impacto
do ruído e de agentes intransigentes.

Contudo, o trabalho pode ser pensado como um único passo num processo de
iterados esforços, teóricos e empíricos, que nos levem à melhor compreensão do
fenômeno \cite{wilensky2015introduction}. Sendo assim, possui várias limitações
cujas resoluções fogem ao seu escopo. A primeira limitação do trabalho é que não
busca modelar o papel da mídia na formação da opinião dos agentes. A mídia pode
ser considerada uma das principais fontes de influência social, mas investigar
como modelar seu efeito envolve compreender outros tipos de mecanismo.

Outra questão importante é referente à regra de interação. Muitos resultados em
influência social, por exemplo os de influência social repulsiva, são na verdade
encontrados em interações em grupo \cite{bramson2017understanding}. Compreender
a diferença da interação entre pares e em grupo, quais as regras de interação e
atualização subjacentes, como traduzir isso em \textit{schedules} em simulações
e qual o impacto sistêmico disso é algo a ser explorado.

A terceira, e talvez principal, limitação do trabalho é a falta de calibração e
validação empírica do modelo. \citeonline{flache2017} argumenta que uma das
principais limitações da literatura em OD é a falta de preocupação com o
confrontamento rigoroso com dados. Dentre os diversos testes possíveis elencados
por eles, consideramos que seria especialmente interessante testar as
macro-predições do modelo com dados de \textit{surveys} (como o ESS).

Uma limitação relacionada com a anterior é que não testamos diferentes
topologias de interação entre os agentes. O modelo foi testado unicamente num
grafo completo, mas um dos principais resultados da área de dinâmicas de opinião
é exatamente a importância da topologia para explicar propriedades macroscópicas
\cite{ball2002physical}. Calibrar a topologia com dados por sua vez é um
desafio metodológico.

Uma limitação mais geral em OD é a apontada por
\citeonline{acemoglu2011opinion}: a literatura costuma focar em propriedades de
longo prazo da dinâmica de opiniões. Muitos fenômenos sociais não são capturados
por esse limite de longo prazo e uma fronteira para a área é quais regras de
interação e atualização melhor caracterizam processos de médio e curto prazo.

Por fim, o modelo teve por \textit{rationale} limitações nos modelos de tomada
de decisão eleitoral e competição partidária. Uma das fronteiras de pesquisa em
ABM é exatamente a construção de modelos multi-nível por meio do acoplamento de
sub-modelos parcialmente autônomos \cite{hjorth2016turtles,head2015evolving}. Um
prospecto animador, portanto, é a possibilidade de criar narrativas
computacionais que liguem as áreas de opinião pública, comportamento eleitoral e
competição partidária, de forma que possamos investigar \textit{feedbacks} entre
os diferentes níveis de ação.



