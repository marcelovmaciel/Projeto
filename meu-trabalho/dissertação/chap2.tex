Nesse capítulo fazemos uma revisão bibliográfica da área de Dinâmicas de
Opinião (OD). Definimos a área, o
que são modelos baseados em agente e quais os constituintes típicos de um modelo
de OD. Depois apresentamos modelos que inspiraram uma gama de modificações e
extensões. Na seção seguinte discutimos algumas questões teóricas referentes à
atualização da opinião dos agentes e concluímos o capítulo com uma discussão de
nossa abordagem.


\section{Definição da Área}

OD é uma área que pode ser definida a partir de 3 elementos. Primeiramente,
sistemas alvo, fenômenos de interesse, em comum, delimitados pela pergunta
central: quais elementos determinam se um grupo de agentes chega ao consenso
sobre algo, ou ao invés disso persistem em discórdia?
\cite{castellano2012social}\footnote{Essa pode ser pensada como a pergunta
  \textit{fundacional} da área \cite{flache2017}.}. Segundo, um conjunto de
modelos que partilham elementos constitutivos, particularmente fazendo uso da
técnica da Modelagem Baseada em Agentes (ABM), e em, alguma medida, de
\textit{insights} e técnicas da Física Estatística \cite{galam1990social}.
Terceiro, uma comunidade de pesquisadores que partilham do interesse no objeto,
fazem uso de referenciais e técnicas compartilhadas e se reconhecem como membros
dessa comunidade.

Na área há a aceitação de um significado amplo e abstrato de opinião como uma
característica de um agente que pode ser mudada com pouco custo
\cite[p.312]{castellano2012social}. Isso permite com que os pesquisadores visem
sistemas alvos tais como voto, ciência, cultura, difusão de tarifas, dentre
outros
\cite{kowalska2013going,martins2015thou,axelrod1997dissemination,galam1990social}.
Essa gama de aplicações está relacionada com a base disciplinar dos
pesquisadores, envolvendo pessoas de áreas como Física, Sociologia, Ciência
Política, Economia, Psicologia Social, dentre outras, o que nos permite
considerar a área como um subgrupo da Sociofísica
\cite{galam1982sociophysics,galam2012sociophysics}.


\section{Modelagem Baseada em Agentes e Dinâmicas de Opinião}

ABMs podem ser definidos como modelos que envolvem agentes discretos, onde
agentes, seus atributos, e possivelmente um ambiente são definidos
algoritmicamente \cite{sayama2015introduction} \footnote{ ABMs costumam ser
  implementados como simulações num computador, embora existam modelos baseados
  em agentes que historicamente não tenham sido diretamente em computadores,
  como os modelos de Schelling e de Sakoda \cite{hegselmann2017thomas}.}.Segundo
\citeonline{de2014agent}, num ABM existem três noções primitivas: os
\textit{atributos}, os \textit{estados} e as \textit{configurações}. Os
atributos dos agentes são o conjunto de propriedades que cada cada agente \(i\)
tem. Os estados dos agentes são os valores de seus atributos num determinado
tempo \(t\). Já as configurações são as coleções de todos os estados dos agentes
num modelo.

ABM é uma técnica flexível: podemos construir modelos metafóricos com
objetivo de auxiliar o desenvolvimento de intuição segundo a elucidação de
princípios; ou de alta-fidelidade, com dezenas de atributos e um
ambiente incluindo casas, escolas, sistemas de transporte, dentre outros, com o
objetivo avaliar contrafactuais próximos a determinados casos concretos
\cite{de2014agent, epstein2006generative}.


Segundo \citeonline[p.430-1]{sayama2015introduction}, ABMs têm as seguintes
propriedades típicas:
\begin{itemize}
\item agentes podem ter estados internos;
\item agentes podem ser espacialmente localizados;
\item agentes podem perceber e interagir com o ambiente;
\item agentes podem interagir segundo regras pré-definidas;
\item agentes podem ser capazes de aprender e adaptar-se;
\item agentes podem interagir com outros agentes;
\item AMBs muitas vezes não tem supervisores/controladores centrais;
  \item ABMs podem produzir comportamentos coletivos não triviais.
  \end{itemize}

  Tendo em vista essas propriedades, ABMs são particularmente úteis para o
  estudo de sistemas complexos \cite{wilensky2015introduction}, dada: sua
  capacidade de incluir redes e espaço; seu potencial de ligar múltiplos
  domínios e de incluir uma maior heterogeneidade de agentes; além de seu foco
  na robustez de resultados \cite{de2014agent,wilensky2015introduction}. ABMs
  são assim capazes de conectar como propriedades macro surgem a partir das
  regras de interação e atributos dados a unidades discretas, os agentes
  \cite{north2007managing}. Ademais, tendo em vista sua flexibilidade,modelos
  baseados em agentes são úteis quando sistemas ou situações de ação são
  complicados demais para serem modelados por meio de modelos matemáticos
  tradicionais \cite{kollman2003computational}. Não por acaso, ABMs são
  amplamente usados em OD \cite{castellano2012social,flache2017}.

  Essas propriedades são ditas emergentes na medida em que caracterizam o
  sistema e não seus componentes. Isto é, propriedades emergentes são aquelas
  que resultam da interação entre as partes do sistema, mas que não seriam a
  priori dedutíveis unicamente das propriedades delas. O conceito, modernamente,
  é relativo tanto ao quadro conceitual quanto às expectativas dos pesquisadores
  sobre quais propriedades sistêmicas poderíamos deduzir a partir das
  propriedades das partes \cite{epstein2006generative}. Em sua forma mais
  simples propriedades emergentes condicionam indiretamente o comportamento dos
  agentes. Se os agentes tiverem consciência dessa propriedade sistêmica temos,
  contudo, um \textit{feedback} direto, ao invés de indireto, entre agentes e
  propriedades sistêmicas, o que \citeonline{squazzoni2008micro} chama de
  emergência de segunda ordem. No nosso trabalho lidamos somente com
  \textit{feedbacks} indiretos, ou emergência de primeira ordem.\todo[color =
  yellow!10]{talvez falar mais disso...}

  Que elementos constituem os modelos de OD ? Podemos delimitar um modelo de
  dinâmicas de opinião da seguinte forma: agentes conectados possuem opiniões
  como variáveis e interagem segundo regras que explicam a mudança ou manutenção
  das opiniões individuais sob efeito da interação com outros agentes ou outras
  fontes (como a mídia) \cite{sirbu2017opinion}. Os agentes num modelo em OD têm
  então: uma \textit{opinião}; uma \textit{estrutura de interação}; e uma
  \textit{regra de atualização} de sua opinião.


  A opinião dos agentes pode ser representada como uma variável ou conjunto de
  variáveis, que por sua vez podem ser discretas ou contínuas. Já a estrutura de
  interação consiste no conjunto de agentes cujas ações e propriedades podem
  afetar a opinião de um agente \(i\) \cite{page2008uncertainty}.

  Podemos dividir a estrutura de interação numa \textit{topologia} de interação e
  numa \textit{regra de interação}. A topologia de interação define quais
  agentes estão conectados com \(i\), e podem, potencialmente, afetá-lo. A regra
  de interação define como \(i\) interage com os agentes desse conjunto(seus
  ``vizinhos''). Em OD as regras de interação definem qual a
  relação que o agente \(i\) tem com seus vizinhos: se interage com um vizinho
  por vez, uma interação em díade, ou com algum subconjunto de seus vizinhos,
  uma interação em grupo. Por fim, a regra de atualização define sob qual regra
  a opinião do agente \(i\) muda do tempo \(t\) para o tempo \(t+1\). 

  

  \section{Modelos Canônicos}

  Adaptações do Modelo de Ising são os modelos mais fundamentais na área. O
  modelo de Ising é um modelo paradigmático da Mecânica Estatística, usado para
  representar o processo de magnetização de materiais\footnote{O modelo de Ising
    é um modelo paradigmático de sistemas com muitas partes interagindo levando
    à uma transição de fase, a uma mudança de comportamento qualitativo do
    sistema. Sendo assim, é aplicado em vários contextos além da sua concepção
    original, como mercados financeiros, sistemas ecológicos, e dinâmicas de
    opinião \cite{sole2011phase}}. Nestas adaptações, variáveis discretas,
  \textit{spins}, com valores $s = \pm 1$, estão localizadas num grafo e têm uma
  tendência a alinhar-se com seus vizinhos: se a maioria tem \(s = + 1 \) o spin
  muda seu valor para \(+1\); se a maioria tem \(s = -1 \) o spin muda seu valor
  para \(-1\); se houver empate o spin muda seu valor com probabilidade
  \(\frac{1}{2}\) \cite{castellano2012social,sole2011phase}. A reinterpretação
  para o contexto de OD é o seguinte: o spin é um agente; sua opinião pode ter
  os valores \(+1\) ou \(-1\); um agente interage com todos seus vizinhos por
  passo de tempo; ele assume a opinião da maioria deles.

  Um modelo parecido com o anterior é o ``Voter'' \cite{holley1975ergodic}. Neste
  cada agente tem uma opinião binária \(\pm\) 1 ; e a cada passo um agente é
  selecionado aleatoriamente e assume a opinião de algum de seus vizinhos.
  Difere do modelo anterior, portanto, na regra de interação (díade ao invés do
  grupo inteiro) e de atualização (assume o valor do vizinho ao invés da maioria
  deles).


  Já no modelo da Regra de Maioria a interação é : a cada
  unidade de tempo um grupo
  de tamanho \textit{r} é selecionado aleatoriamente e todos os agentes mudam
  sua opinião para a opinião da maioria do grupo
  \cite{galam1990social,galam2012sociophysics}. O tamanho \textit{r} pode ser
  fixo ou ser tirado de alguma distribuição a cada passo. Se \textit{r} for par
  podem ocorrer empates nos grupos, de forma que ou o grupo escolhe uma das
  opiniões com probabilidade \(\frac{1}{2}\), ou introduz-se um viés, e toda vez
  que houver empate o grupo muda para uma das opiniões
  \cite{galam2012sociophysics, galam1986majority}.

  O Modelo Sznajd também é bastante discutido na literatura
  \cite{sznajd2000opinion, sirbu2017opinion,castellano2012social}. Em sua versão
  original, e mais simples, cada agente têm exatamente dois vizinhos, em uma
  grade unidimensional. A cada passo um par $ij$ de vizinhos é selecionado e se
  sua opinião for igual os outros vizinhos de \(i\) e \(j\) mudam a opinião para
  a opinião de convergência. Se eles discordarem, \(i\) adota a opinião do outro
  vizinho, e \(j\) faz o mesmo.

  Todos os modelos até agora representaram opiniões como uma variável que pode
  tomar valores binários. Além disso a regra de atualização dos modelos
  pressupõe uma interação assimilativa: indivíduos conectados por meio de uma
  relação estrutural influenciam uns aos outros em direção à diminuição da
  diferença de suas opiniões \cite{flache2017}. O modelo de
  \citeonline{axelrod1997dissemination} difere em ambos os aspectos. Cada agente
  tem por opinião um vetor $F$ de componentes $(\sigma_1 , \ldots, \sigma_f)$
  \cite{klemm2003role}. Esses $\sigma_i$ podem tomar valores inteiros de 0 a 9. Os
  componentes são as características culturais dos agentes e seus possíveis
  valores são seus traços culturais \cite{gomes2014}. O modelo considera
  interação entre pares de vizinhos, os quais interagem com uma probabilidade
  proporcional ao número de traços que têm igual. Isso significa que se \(i\)
  tem uma opinião igual a 82330 e seu vizinho \(j\) tem uma opinião 67730 eles
  têm \(40 \%\) de interagirem. Se eles interagirem, \(i\) troca um dos traços
  em que difere \(por\) j um dos traços de \(j\)\cite{axelrod1997dissemination}.
  Nesse modelo pessoas similares têm uma probabilidade maior de se interagirem
  do que pessoas distintas, mas uma vez que a interação ocorre elas ficam mais
  parecidas. Sendo assim, o modelo de Axelrod pode ser considerado um modelo de
  assimilação enviesada: só indivíduos suficientemente similares podem
  influenciar uns aos outros na redução de suas diferenças\footnote{Um terceiro
    tipo de modelo elencado por \citeonline{flache2017}, além dos modelos de
    influência social assimilativa e modelos com influência enviesada, são os
    modelos de influência repulsiva: quando agentes são muito distintos em
    opinião a interação pode levá-los a um aumento dessa diferença, isto é,
    tornam-se mais distantes.} \cite{flache2017}.


  Um outro modelo de assimilação enviesada é o Modelo de Deffuant-Weisbuch
  \cite{deffuant2000mixing}. Nele cada agente \(i\) tem opinião inicial \( o_i \in
  [0,1]\). Dois agentes são escolhidos aleatoriamente, e \(i\) é influenciado
  por \(j\) se \(| o_i - o_j| < \epsilon\). Se isso ocorrer suas opiniões se aproximam
  de acordo com um parâmetro $0 < \mu< \leq 0.5$, de forma que: $o_{i,t+1} = o_{it} +
  \mu(o_{jt} - o_{it})$. Esse modelo é particularmente relevante para o presente
  trabalho, por duas razões: a opinião é contínua, assim como a representação
  das ``utilidades'' dos agentes em Teoria Política Espacial \footnote{Na
    verdade, ligar a literatura de OD com funções de utilidade em economia e com
    a noção geométrica de política é a justificativa dada por
    \citeonline{deffuant2000mixing} para considerar opiniões como contínuas.}; e
  \(\epsilon\) pode ser interpretado como parte da regra de atualização, o que faz
  modelo os agentes tenham viés de confirmação \cite{flache2017}.

Quando \(\epsilon\) é interpretado como parte da regra de interação temos o princípio
da homofilia: padrões estruturais de interação social levam pessoas a ter maior
probabilidade de interagirem com pessoas similares a elas
\cite{mcpherson2001birds}. Quando \(\epsilon\) é interpretado como parte da regra de
atualização temos o fenômeno do \textit{viés de confirmação}: a tendência das
pessoas de dar maior peso a informações que confirmem suas crenças anteriores
\cite{nickerson1998confirmation}.

\citeonline{huckfeldt2005patterns} argumenta que indivíduos escolhem redes de
discussão com razões distintas às políticas (como interesses profissionais e
hobbies) e acabam interagindo com indivíduos cujas filiações partidárias
distintas. Desta forma, o papel da homofilia em política é atenuado. Já o
\textit{vies partidário} dos cidadãos, o viés de confirmação no tocante a
questões políticas , é um resultado estabelecido na literatura em opinião
pública e psicologia política, e imprescindível para a modelagem generativa de
opinião pública \cite{bartels2002beyond, flynn2017nature,
  lodge2013rationalizing}.


\section{Regra de Atualização e Processamento de Informação}


Como lembra Dirk Helbing, não existe uma única forma de modelar agentes
interagindo em sistemas sociais complexos \cite{helbing2010pluralistic}. Os
modelos clássicos em OD têm uma abordagem que Helbing chama de
\textit{fisicalista}: abstraem as interações sociais ao ponto delas poderem ser
estudadas como um modelo de ``partículas''. Em OD essa abordagem é refletida na
forma como se modela a regra de atualização: abstrai-se o processamento de
informação, a cognição dos agentes. Isso, como frisa Helbing, não é uma falha
dos modelos. Paul Ormerod defende que o \textit{null model} em sistemas sociais
complexos deveria ser o de um \textit{zero intelligence actor}, pois a
complexidade dos sistemas nos permitiria modelar os agentes ``como se fossem''
átomos \cite{ormerod2008can, bentley2012agents}.

Não obstante, um conjunto de trabalhos em OD tem buscado abrir a ``caixa-preta''
da cognição dos agentes e tratam a atualização de opinião como resultado de um
processamento de informação explicitamente modelado \cite{flache2017,
  jager2017}. O modelo Polias de \citeonline{brousmiche2016beliefs}, o modelo
Innomind de \citeonline{schroder2017modeling} e o modelo Lodge-Taber
\cite{kim2010computational,kim2011model} de processamento dual e raciocínio
motivado são exemplos dessa tendência.

Se pensarmos num espectro possível de abordagens para a cognição dos agentes,
ilustrado na Figura \ref{fig4} , esses modelos estão posicionados no extremo oposto aos
modelos fisicalistas. Enquanto os modelos fisicalistas abstraem totalmente o que
se passa na cabeça dos agentes os modelos (neuro)cognitivos buscam representar a
arquitetura cognitiva que alicerça as atitudes e crenças
deles \cite{kim2010computational}.

\begin{figure}[H]
  \centering
  \includegraphics[width = \textwidth, height = 3cm]{ims/line.pdf}
  \caption{Espectro de abordagens no tocante à cognição dos agentes}
  \fonte{O autor.}
  \label{fig4}
\end{figure}

Modelos cognitivamente ``densos'' permitem que analisemos como processos de
influência social estão micro-fundamentados em processos mentais subjacentes, e
são um fronte dentre os trabalhos que buscam aumentar o realismo das simulações
sociais \cite{jager2017,epstein2014agent_zero, conte2013minding}. Contudo, como
ressalta Jonathan Bendor, na medida em que a Ciência Política busca modelar
macrofenômenos ela ``deve ser mais implacável em relação aos micropressupostos
do que microcampos relacionados (como ciência cognitiva)''
\cite[p.45]{bendor2010bounded}. Quanto mais complicados nossos modelos menor
controle temos sobre qual o elemento responsável pelo seus resultados, e mais
dados precisamos para a sua calibração e validação \cite{de2005computational,
  bendor2010bounded}. Como argumentado por \citeonline{zaller1992nature}, a
estratégia metodológica para modelar a opinião pública, um fenômeno social de
larga escala, envolve incorporar no modelo somente os aspectos do processamento
de informação que têm relevância para a compreensão das dinâmicas do fenômeno,
ao invés de buscar desenvolver modelos que aproximam o mais próximo o possível
os detalhes da mente humana\footnote{No contexto da análise institucional um
  argumento equivalente é feito por \citeonline{ostrom1990governing}.}.


\citeonline{martins2012bayesian} apresenta um \textit{framework} para modelar
Dinâmicas de Opinião que é cognitivamente mais
``denso'' que os modelos fisicalistas, mas sem buscar modelar as bases
neurocognitivas de processamento de informação. O \textit{framework} está
fundamentado no uso da inferência bayesiana como base da regra de atualização
dos agentes. Embora seja bem documentado que as pessoas não seguem fielmente o
princípio da racionalidade bayesiana, apresentado no primeiro capítulo, um
conjunto de trabalhos em psicologia e ciência cognitiva vêm, nos últimos anos,
defendendo a possibilidade de que sejamos ``bayesianos
imperfeitos''\cite{griffiths2006optimal,fujikawa2007perfect,baker2017rational,
  gintis2016individuality}. Usar um framework bayesiano é, desta forma, uma
aproximação e permite a construção de modelos de dinâmicas de opinião de uma
forma fundamentada num princípio comum, algo particularmente relevante numa área
em que há a proliferação de modelos ad-hoc \cite{flache2017,jager2017}.

 \citeonline[p.214]{martins2012bayesian} oferece o seguinte passo a passo
para a construção de um modelo de dinâmicas de opinião com base num
\textit{framework} bayesiano:

\begin{enumerate}
\item Identificar uma questão sob debate e chamá-la de $x$. \(x\) pode ser
  discreto ou contínuo.
\item cada agente \(i\) tem uma opinião subjetiva sobre $x$ e essa opinião é
  representada pela distribuição de probabilidade $f_i(x)$ .
\item Ocorre comunicação : a comunicação é a declaração de um valor
  $ A_j$ pelo agente $j$ de tal forma que $A_j[f]$ é um funcional de
  $f_j(x)$.
\item Os agentes tem que ter em sua mente  uma relação entre o
  verdadeiro valor entre $x$ e o valor declarado $A_j$. Isso é dado
  pela distribuição de probabilidade $P(A_j|x)$.
\item Dado o prior $f_i(x)$ a opinião posterior $f_i(x|A_j)$ é dada
  por $A_i[f_i(x|A_j)]$ que é a nova opinião de $i$ .
\end{enumerate}





A aplicação desse \textit{framework} no caso de opiniões contínuas incorpora o
mecanismo de viés de confirmação presente no modelo Deffuant-Weisbuch, encontra
resultados qualitativamente semelhantes, e ainda tem resultados adicionais
\cite{martins2009bayesian}. Dado que o modelo representa a atualização de uma
variável contínua, incorpora viés de confirmação e fundamenta a regra de
atualização num framework bayesiano de processamento de informação ele será a
base do nosso trabalho.

Nele os agentes interagem num grafo completo e a regra de interação é uma díade
interagindo a cada passo de tempo. A opinião inicial dos agentes é dada por uma
probabilidade subjetiva:


\begin{align}
f_i(\theta) = \frac{1}{\sqrt{2 \pi} \sigma_i} e^{-
  \frac{(\theta - o_i )^2}{2 \sigma_i}}
  \end{align}

  Onde \(o_i = E_i[\theta] \), de forma que \(E_i\) é o valor esperado que o agente
  associa a \(\theta\). Para incluir viés de confirmação na regra de atualização,
  \citeonline{martins2009bayesian} introduz uma probabilidade \(p\) que o outro
  agente saiba algo sobre $\theta$ isto é, que tenha uma opinião plausível e que vale
  a pena ser levada em conta ; e vai ter uma chance \(1 - p\) que o outro agente
  não tenha informação sobre $\theta$ de forma que a verossimilhança de $j$ estar
  correto é dado por $ f(o_j|\theta) = p N(\theta,\sigma_j^2) + (1-p)U(0,1)$.

  Temos então que a distribuição da nova opinião é dada por uma mistura de duas
  normais com médias diferentes, que é proporcional à  multiplicação da
  verossimilhança e a distribuição a priori normal de $i$:
  
  \begin{align}
    f(\theta | o_j)
    \propto 
    p
    e^
    {-(\frac{1}{2\sigma_i^2})
    [(\theta - o_i)^2
    +
    (o_j - \theta )^2
    ]}
    +
    (1-p)
    e^{-\frac{(o_i - o_j)^2}{(2 \sigma_i^2)}}
  \end{align}

 Se calcularmos o $E_i(\theta)$ da expressão anterior temos a nova opinião:
  \begin{align}
    o_i(t+1)
    =
    p
    \frac{o_i(t) + o_j(t)}{2}
    +
    (1-p^*)o_i(t)
  \end{align}

  Onde:
  \begin{align}
    p^*
    =
    \frac{
      p \frac{1}{\sqrt{2 \pi} \sigma_i}
      e^{(- \frac{o_i (t) - o_j (t))^2}{2 \sigma_i^2})}
    }{
      p
      \frac{1}{\sqrt{2 \pi} \sigma_i}
    e^{(- \frac{o_i (t) - o_j (t))^2}{2 \sigma_i^2})}
    +
    (1 - p)
    }
  \end{align}

  Importante notar que embora o modelo use a Regra de Bayes para derivar como o
  agente atualiza sua opinião nele o agente não é perfeitamente racional. A
  regra de bayes foi usada para criar uma regra de atualização plausível, mas a
  verossimilhança de um agente perfeitamente racional envolveria considerar
  toda a informação possivelmente relevante
  , por exemplo, se \(j\) tem um interesse estratégico em declarar um
  determinado \(o_j\) ou como \(o_j\) é resultado da interação de j com outros
  agentes. Ademais, suponha que \(i\) e \(j\) interagem num tempo \(t_a\) e
  depois num tempo \(t_b\). Um agente perfeitamente racional corrigiria pela
  repetição da interação \cite{acemoglu2011opinion}, já no modelo de
  \citeonline{martins2009bayesian} o agente age segundo a heurística dada pela
  Equação 2.4. Nele, portanto, os agentes são ``imperfeitamente'' bayesianos.
  Com essa regra de atualização o modelo já recupera os resultados dos modelos
  de confiança limitada. Contudo, também é possível considerar o caso em que a
  incerteza dos agentes é atualizada:

      \begin{align}
    \sigma_i^2(t+1)
    =
    \sigma_i^2(t)
    (1 - \frac{p^*}{2})
    +
    p^*
    (1-p^*)
    (\frac{o_i(t)-o_j(t)}{2})^2
      \end{align}

      Como demonstra \citeonline{martins2009bayesian} essa equação faz com que
      os agentes fiquem mais certos de suas opiniões. Isso é relevante do ponto
      de vista da modelagem generativa de opinião pública.
      \citeonline{kuklinski2000misinformation} encontram que um grande
      percentual,em média \( 60 \%\), dos respondentes em um
      \textit{survey} por telefone sobre questões factuais da agenda política
      americana da época são confiantes em suas crenças. Ademais, encontram uma
      correlação entre confiança e partidarismo. É interessante, portanto,
      incorporar não só a dinâmica da opinião dos agentes, mas também o papel da
      incerteza quanto as crenças nessa dinâmica \footnote{Para um trabalho em
        OD que lida com incerteza, mas de maneira ``fisicalista'' ver
        \citeonline{deffuant2002can}.}.


