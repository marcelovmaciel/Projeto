O modelo proposto no trabalho está fundamentado no diálogo entre a Teoria
Política Espacial, e a área de Dinâmicas de Opinião. A teoria geométrica de
política modela as preferências dos agentes como relações em um espaço contínuo,
as quais, em ambientes macro, são construídas por meio da agregação, ou sumário,
das atitudes, crenças, posicionamentos, ou simplesmente opiniões dos agentes em
diferentes questões (\textit{issues}). As preferências dos agentes numa dimensão
são, assim, o sumário de um \textit{perfil ideológico} do agente sobre questões.
Vamos assumir, portanto, que a preferência do agente é centrada num ponto ideal
derivado das opiniões dele nessas questões.

Assumimos,portanto, que as preferências dos agentes na dimensão são de
pico-único, o pressuposto modal, desde \citeonline{black1958theory}, na
literatura em modelos fortemente espaciais. Para gerar a distribuição de
preferências, contudo, não precisamos especificar qual a função de utilidade
centrada nesse ponto ideal, já que não é interesse do modelo incluir a tomada de
decisão que o pressuporia, por exemplo a escolha de um candidato.

Sendo assim, cada agente vai ter por atributo um perfil ideológico \(I\), onde
\(I_i = (f_i(\theta_1), \ldots, f_i(\theta_n)) \). Os elementos de \(I\) são as crenças dos
agentes em cada questão. Seguindo \citeonline{martins2012bayesian}, vamos pressupor
que os agentes têm uma probabilidade subjetiva sobre cada questão \(\theta\), e uma
opinião \( o_i = E_i[\theta]\) e incerteza \( \sigma_i^2 = E[\sigma^2] - E_ i[\theta]^2\)
associados. O ponto ideal do agente vai ser dado pela média aritmética das
opiniões dele em \(I\)\footnote{Em trabalhos futuros, é interessante pensar o
  ponto ideal como a média ponderada. Adicionar peso a cada questão é uma forma
  de modelar a \textit{saliência} da questão para os agentes. A saliência de uma
  questão para um indivíduo é a importância que ele dá a ela
  \cite{munger2015choosing}.} A regra de interação é em díades, os agentes
interagem com um vizinho por vez. Quando os agentes interagem \(i\) atualiza sua
opinião e incerteza em alguma\footnote{Qual questão vão ``debater'' vai ser por
  meio de um sorteio sem viés. Em trabalhos futuros é interessante considerar um
  viés, o que representaria saliência no sentido dado por
  \citeonline{zaller1992simple}: qual questão os agentes estão dando atenção,
  isto é, qual questão está mais acessível na memória deles.} questão segundo as
equações 2.3 e 2.5.

Vamos considerar também que os agentes reconsideram sua opinião e certeza sobre
as questões segundo uma probabilidade \(\rho\). Do ponto de vista teórico, estamos
considerando a possibilidade de fatores não relacionados à influência social
levarem o agente a mudar seu posicionamento sobre questões \cite{flache2017,
  lorenz2017modeling}. Do ponto de vista metodológico, \cite{macy2015signal}
argumenta que adicionar ruído contribui para a robustez das conclusões tiradas
por modelos teóricos.





