\section{Entidades e Processos}

O trabalho tem por propósito explorar a emergência de distribuições de
preferências fundamentando-se no diálogo entre a Teoria Política Espacial e a
área de Dinâmicas de Opinião. Não temos por propósito um modelo que seja
preditivo, mas que capture microfundamentos relevantes, como viés de confirmação
e confiança nas crenças, e gere distribuições de preferência plausíveis, no
sentido delimitado no Capítulo 1. O trabalho propõe, portanto, um modelo que
abra a caixa-preta do processo de socialização que leva a cristalização de
direcionamentos, ou posicionamentos, ideológicos.

A Teoria Geométrica de política modela as preferências dos agentes como relações
em um espaço contínuo, as quais, em ambientes macro, são construídas por meio da
agregação das atitudes, crenças, posicionamentos, ou simplesmente opiniões dos
agentes em diferentes questões (\textit{issues}). As preferências dos agentes
numa dimensão são, assim, o sumário de um \textit{perfil ideológico} do agente
sobre questões.

Para gerar a distribuição de pontos ideais, contudo, não precisamos especificar
qual a função de utilidade centrada nele, já que não é interesse do trabalho
modelar a tomada de decisão que o pressuporia, por exemplo a escolha de um
candidato. Como discutido no Capítulo 1, é possível atribuir diferentes funções
de utilidade aos agentes, mas o pressuposto modal é que a função vai ter um
máximo e será simétrica \cite{eguia2013spatial, carroll2013structure}. Vamos
fazer uso desse pressuposto e modelar somente a emergência do ponto preferido a
partir da interação com outros agentes em várias questões. Como não é um modelo
de tomada de decisão, mas sim um modelo de surgimento de posicionamentos
ideológicos, não é necessário adicionar mais um elemento, funções de utilidade,
que simplesmente alargaria o espaço de parâmetros e não seria utilizado na
simulação. Contudo, modelos futuros que busquem ligar, por exemplo, a
distribuição de preferências com a escolha de candidatos/partidos poderiam fazer
essa atribuição.

%Assumimos, portanto, que as preferências dos
%agentes na dimensão são de pico-único, o pressuposto modal, desde
%\citeonline{black1958theory}, na literatura em modelos fortemente espaciais, e
%em trabalhos empíricos em estimação de pontos ideais \cite{carroll2013structure,
% armstrong2014analyzing, schofield1998nash}.

Pensar os agentes como tendo ideais derivados de posicionamentos em outras
questões tem por base dois fundamentos. O primeiro é que esse elemento a mais, em
comparativo aos modelos de \citeonline{deffuant2000mixing} e de
\citeonline{martins2012bayesian}, nos permite ser mais condizentes com a
literatura discutida no Capítulo 1 em contraposição à equiparação do ponto ideal
a uma opinião. Isto é, os pontos ideais dos agentes vão mudar ao longo da
simulação, mas isso ocorre devido à mudança nas suas crenças. É uma
mudança assim indireta e condizente com a noção de que a ideologia do agente é
um atributo extrínseco. O segundo fundamento é que essa modificação tem por
implicação a capacidade de adicionar outros elementos à dinâmica do modelo.


Sendo assim, cada agente vai ter por atributo um perfil ideológico \(I\), onde
\(I_i = (f_i(\theta_1), \ldots, f_i(\theta_n)) \). Os elementos de \(I\) são as crenças dos
agentes em cada questão. Seguindo \citeonline{martins2012bayesian}, vamos pressupor
que os agentes têm uma probabilidade subjetiva sobre cada questão \(\theta\), e uma
opinião \( o_i = E_i[\theta]\) e incerteza \( \sigma_i^2 = E[\sigma^2] - E_ i[\theta]^2\)
associados. O ponto ideal \(x\) do agente vai ser a média aritmética das opiniões dele em
\(I\)\footnote{Em trabalhos futuros, é interessante pensar o ponto ideal como a
  média ponderada. Adicionar peso a cada questão é uma forma de modelar a
 importância que o indivíduo dá a ela. Essa é uma implicação possível ao
 pensarmos o agente em termos de perfil ideológico ao invés de uma única
 opinião.}. Do ponto de vista da implementação esse atributo é reduzido a um
conjunto de pares \(I_i = ((o_1,\sigma_1^2), \ldots, (o_n, \sigma_n^2) )\), retirados de
distribuições uniformes independentes, e o posicionamento ideológico do agente é
dado por \(x_i = \sum_{k=1}^{n} o_k\).

A regra de interação é em díades. A cada passo no tempo um dos agentes vai ser
escolhido e vai interagir com um de seus vizinhos, a princípio num grafo
completo. A interação é assim assíncrona (os agentes atualizam seus atributos em
momentos distintos) e sequencial (um agente atualiza por vez)
\cite{wilensky2015introduction}. A dependência dos resultados em relação ao
número de agentes e de crenças vai ser explorada.

Quando os agentes interagem \(i\) atualiza sua opinião e incerteza em
alguma\footnote{Qual questão vão ``debater'' vai ser definido por meio de um
  sorteio sem viés. Uma outra implicação possível, a ser adicionada em trabalhos
  futuros, é considerar um viés nessa seleção, o que representaria saliência no
  sentido dado por \citeonline{zaller1992simple}: qual questão os agentes estão
  dando atenção, isto é, qual questão está mais acessível na memória deles.}
questão segundo as equações 2.3 e 2.5. Vamos considerar duas variantes de \(p\):
na primeira ele vai ser uma variável global \(0 < p < 1 \), a mesma para todos
os agentes, assim como em \citeonline{martins2009bayesian}; na segunda,
consideramos a possibilidade de que os agentes levem em conta ou não a opinião
do vizinho em uma questão em particular a partir do conhecimento do ponto ideal
do outro agente, de forma que: \(p_{ij}(t) = 1 - |x_i(t) - x_j(t)|\). Isto é,
nesse \(p\) alternativo o agente \(i\) vai considerar a verossimilhança de um
valor declarado pelo seu vizinho \(j\) segundo um \(p\) derivado da média do
posicionamento ideológico de \(j\). Por exemplo, suponha que o agente \(j\) é um
eleitor conservador. O agente \(i\) vai levar
em conta essa informação geral sobre \(j\) para considerar a probabilidade do
agente \(j\) estar falando algo correto em questões particulares.

Os agentes também vão reconsiderar suas opiniões e certezas sobre as questões
segundo uma probabilidade \(\rho\). Do ponto de vista teórico, estamos considerando
a possibilidade de fatores não relacionados à influência social levarem o agente
a mudar seu posicionamento sobre questões \cite{flache2017, lorenz2017modeling}.
Do ponto de vista metodológico, \citeonline{macy2015signal} argumenta que
adicionar ruído contribui para a robustez dos resultados de modelos
teóricos.\todo{discutir mais isso do ruido} 

\section{Parâmetros-Chave}

Os parâmetros-chave para configuração do modelo, cujos valores seguem
\citeonline{martins2008continuous}, \citeonline{deffuant2000mixing} e \citeonline{lorenz2017modeling},  são:
\begin{itemize}
\item A população de \(500 < N < 10000\) agentes;
\item O número de questões \(1 \leq n \leq 10\); 
\item As opiniões \(0.0< o_i< 1.0\) dos agentes
  \begin{itemize}
  \item A opinião dos agentes no tempo \(t = 0\) é retirada de uma distribuição
    uniforme;
  \end{itemize}
\item As incertezas \(0.01 \leq \sigma_i \leq 0.5\);
  \begin{itemize}
  \item A incerteza é, na condição inicial, a mesma para todos os agentes;
  \item Vamos considerar versões em que os agentes atualizam as incertezas e que
    não atualizam.
  \end{itemize}

\item O parâmetro de confiança \(0.1 \leq p \leq 0.99\);
  \begin{itemize}
  \item Vamos considerar variantes em que o \(p\) é global ou em que o \(p\) é
    calculado para cada interação;
  \end{itemize}
  
\item A probabilidade de reconsideração \(0.05 \leq \rho  \leq 0.5\);
  \begin{itemize}
  \item Vamos considerar casos com e sem ruído.
  \end{itemize}
\end{itemize}


\section{Metodologia de Análise}
























%%% Local Variables:
%%% mode: latex
%%% TeX-master: "master"
%%% End:
