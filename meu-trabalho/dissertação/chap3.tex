

Tendo em vista a possibilidade de entrelaçamento entre opiniões e preferências,
vamos seguir a estratégia de modelagem de \citeonline{lorenz2017modeling}: usar
um modelo de dinâmicas de opinião para derivar os pontos ideais. Nisso vamos
fazer uso da ligação entre preferências extrínsecas e crenças/opiniões e fazer
um modelo que dinamiza as preferências dos agentes, nos retendo aso seus pontos ideais
\footnote{Não vamos, assim, estabelecer uma forma funcional específica para a
  função de utilidade dos eleitores. Pressupõem-se, porém, que sejam aproximadas
  por alguma das funções de utilidade tradicionalmente usadas em política e
  apresentadas no Capítulo 1. \citeonline{carroll2013structure} considera que,
  no congresso americano, os legisladores têm uma função de utilidade melhor
  aproximada por uma Normal do que por uma função quadrática.}, a partir da
mudança em suas opiniões \footnote{As quais, como discutido mais a frente, vão
  ser representadas por probabilidades subjetivas.}.

\citeonline{lorenz2017modeling} pode ser, portanto, visto como uma primeira
aproximação na criação de modelos em que os pontos ideais dos eleitores não são
fixos. O modelo baseia-se no modelo de \citeonline{deffuant2000mixing}, o qual é
um modelo canônico na área de Sociofísica \cite{galam1982sociophysics}. O modelo
tornou-se canônico ao, de maneira simples, incorporar à modelagem de dinâmicas
de opinião um elemento importante na interação dos agentes: o fato de agentes
ideologicamente distantes muitas vezes se recusarem a trocar informação, em
decorrência da distância ideológica. Nisso, o modelo foi mais um passo de
aumento da fidelidade dos modelos de dinâmicas de opiniões, com o objetivo de
aumentar sua capacidade  explicativa\footnote{Como lembra \citeonline{de2014agent}, a busca por
  fidelidade só é justificada se tiver como consequência uma maior capacidade
  explicativa, devendo ser de outra maneira evitada. Se um elemento vai
  contribuir ou não em \textit{insights} é, contudo, ainda muito vago, e uma
  lacuna metodológica a ser superada
  \cite{ragan2010embarrassment}.}. 

Uma preocupação da literatura em OD tem sido, assim, em ir além e garantir um
maior realismo cognitivo dos agentes \cite{duggins2014psychologically,homer2013complex,abrica2017effects}. Nisso a
área tem feito um empreendimento semelhante ao de
\citeonline{epstein2014agent_zero} de dar uma melhor fundamentação
(neuro)cognitiva aos modelos que usam a metodologia da Modelagem Baseada em
Agentes, dado que ela nos permite incorporar esses elementos sem uma grande
perda de tratabilidade. Contudo, um princípio básico da modelagem é a
parcimônia, e é questionável em que medida estamos tendo uma alavancagem
explicativa na área de dinâmicas de opinião ao complicarmos demais a cognição
dos agentes \cite{lave1993introduction,sznajd2014person}.

Essa preocupação em balancear plausibilidade, capacidade explicativa, parcimônia
e tratabilidade também coloca em cheque o uso de modelos derivados da Teoria da
Decisão, conhecidos como Bayesianos, e tipicamente usados na Economia
\cite{acemoglu2011opinion}. \citeonline{acemoglu2011opinion} considera que
modelos de OD com agentes Bayesianos pressupõem  muito tanto em relação a
estrutura da opinião dos agentes, exigindo que tenham um ``modelo'' do mundo,
quanto em relação à sua capacidade ``computacional'', ou de processamento de
informação, isto é, de calcular o \textit{posterior} num ambiente com muita
informação e em constante mudança. Seriam mais úteis, portanto, como
contrafactuais extremos ou \textit{benchmark} para comparação com modelos com
agentes mais plausíveis.

Se nem agentes hiper-racionais, nem agentes com excesso de fundamentação
(neuro)-cognitiva são a resposta às limitações dos modelo em OD, nos mantemos
usando modelos  com agentes que mudam de opinião segundo alguma
heurística, como imitação ou rateamento, típicos da literatura em Sociofísica,
mas com pouca fundamentação empírica e comportamental
\cite{acemoglu2011opinion}?

Se temos por objetivo modelos explicativos/preditivos\footnote{Estamos aqui nos
  retendo a modelos que propositalmente buscam fidelidade com sistemas
  concretos, com objetivos explicativos/preditivos. Existe uma gama de outros
  usos, positivos e normativos, para modelos teóricos. Para discussões sobre
  usos de modelos teóricos, particularmente em Ciência Política, ver:
  \citeonline{clarke2012model} e \citeonline{johnson2014models}.} é interessante
que o uso de agentes que seguem heurísticas se coadune com a fundamentação
empírica, mesmo que seja uma calibração informal \cite{flache2017}.\footnote{Não
  necessariamente, contudo, o teste empírico. Como argumenta
  \citeonline{clarke2012model}, não necessariamente modelos empíricos devem ser
  apresentados com um modelo teórico acoplado e vice-versa. Porém, cremos que se
  o propósito do modelo é explicativo  é relevante que tenha  ao menos uma calibração
  informal \cite{flache2017}, e em algum momento seja confrontado, validado,
  com os dados. Em particular, em comparação com outros modelos concorrentes
  \cite{clarke2012model}. } Uma alternativa é o \textit{framework} proposto por
\citeonline{martins2012bayesian} que é inspirado nos modelos Bayesianos, mas não
pressupõe agentes totalmente racionais, modelando-os com uma capacidade de
processamento de informação limitada\footnote{Diferentemente dos modelos
  bayesianos em que os agentes usam toda a informação disponível para tomar sua
  decisão \cite{jackman2009bayesian}}. Um conjunto de trabalhos em psicologia e
ciência cognitiva vêm, nos últimos anos, defendendo a possibilidade de que o
cérebro humano possa ser pensado como um mecanismo de teste de hipóteses
imperfeito\footnote{Essa imperfeição, contudo, teria um fundamento evolucionário
  \cite{price2016hierarchical,martins2005adaptive}.}
\cite{hohwy2013predictive,sanborn2016bayesian}. Inspirar-se na Teoria da
Decisão, a qual lida tradicionalmente com o problema da inferência, pode ser,
portanto, uma boa estratégia na tentativa de modelar dinâmicas de opinião com
agentes ``imperfeitamente bayesianos''
\cite{griffiths2006optimal,fujikawa2007perfect,baker2017rational}.

