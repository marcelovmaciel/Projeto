
\section{Entidades e Processos}

O trabalho tem por propósito explorar a emergência de distribuições de
preferências fundamentando-se no diálogo entre a Teoria Política Espacial e a
área de Dinâmicas de Opinião. Não temos por propósito um modelo que seja
preditivo, mas que capture microfundamentos relevantes, como viés de confirmação
e confiança nas crenças, e gere distribuições de preferência plausíveis, no
sentido delimitado no Capítulo 1.

A Teoria Geométrica de política modela as preferências dos agentes como relações
em um espaço contínuo, as quais, em ambientes macro, são construídas por meio da
agregação das atitudes, crenças, posicionamentos, ou simplesmente opiniões dos
agentes em diferentes questões (\textit{issues}). As preferências dos agentes
numa dimensão são, assim, o sumário de um \textit{perfil ideológico} do agente
sobre questões.

Vamos assumir que a preferência do agente é centrada num ponto ideal derivado
das opiniões dele nessas questões. Assumimos, portanto, que as preferências dos
agentes na dimensão são de pico-único, o pressuposto modal, desde
\citeonline{black1958theory}, na literatura em modelos fortemente espaciais, e
em trabalhos empíricos em estimação de pontos ideais \cite{carroll2013structure,
  armstrong2014analyzing, schofield1998nash}. Para gerar a distribuição de
pontos ideais, contudo, não precisamos especificar qual a função de utilidade
centrada nele, já que não é interesse do trabalho modelar a tomada de decisão
que o pressuporia, por exemplo a escolha de um candidato.

Pensar os agentes como tendo ideais derivados de posicionamentos em outras
questões tem por base dois fundamentos. Um fundamento realista: esse elemento a
mais, em comparativo aos modelos de \citeonline{deffuant2000mixing} e de
\citeonline{martins2012bayesian}, nos permite ser mais condizentes com a
literatura discutida no Capítulo 1 em contraposição à equiparação do ponto ideal
a uma opinião. Isto é, os pontos ideais dos agentes vão mudar ao longo da
simulação, mas isso ocorre devido à mudança nas suas crenças em questões. É uma
mudança assim indireta e condizente com a noção de que a ideologia do agente é
um atributo extrínseco. Ademais temos um fundamento instrumentalista: essa
modificação tem por implicação a capacidade de adicionar outros elementos à
dinâmica do modelo \footnote{Para a distinção entre instrumentalismo e realismo
  ver \citeonline{caldwell1980critique}.}.


\todo[inline, color = yellow!10]{ O que eu não estou conseguindo explicar é que
  a ideologia é um sumário em posições e seguindo o que discuti no chap1 as
  pessoas mudam de opinião a partir de discussoes sobre temas particulares,
  aborto, armas, segurança, economia,  E NÃO uma discussao digamos esquerda x
  direita, liberal x conservador. É nesse sentido que o  modelo é mais
  realista... Ninguem vira mais de esquerda pq teve um debate continuo se ser
  mais ou menos de esquerda é o correto, mas a partir do debate nas questões
  particulares que depois de agregadas dizem se a pessoa é mais de esquerda ou
  mais de direita.... }


Sendo assim, cada agente vai ter por atributo um perfil ideológico \(I\), onde
\(I_i = (f_i(\theta_1), \ldots, f_i(\theta_n)) \). Os elementos de \(I\) são as crenças dos
agentes em cada questão. Seguindo \citeonline{martins2012bayesian}, vamos pressupor
que os agentes têm uma probabilidade subjetiva sobre cada questão \(\theta\), e uma
opinião \( o_i = E_i[\theta]\) e incerteza \( \sigma_i^2 = E[\sigma^2] - E_ i[\theta]^2\)
associados.O ponto ideal \(x\) do agente vai ser a média aritmética das opiniões dele em
\(I\)\footnote{Em trabalhos futuros, é interessante pensar o ponto ideal como a
  média ponderada. Adicionar peso a cada questão é uma forma de modelar a
 importância que o indivíduo dá a ela. Essa é uma implicação possível ao
 pensarmos o agente em termos de perfil ideológico ao invés de uma única opinião.}.

A regra de interação é em díades. A cada passo no tempo um dos agentes vai ser
escolhido e vai interagir com um de seus vizinhos, a princípio num grafo
completo. A interação é assim assíncrona (os agentes atualizam seus atributos em
momentos distintos) e sequencial (um agente atualiza por vez)
\cite{wilensky2015introduction}. A dependência dos resultados em relação ao
número de agentes e de questões que os agentes têm em mente vai ser explorada.

Quando os agentes interagem \(i\) atualiza sua opinião e incerteza em
alguma\footnote{Qual questão vão ``debater'' vai ser definido por meio de um
  sorteio sem viés. Uma outra implicação possível, a ser adicionada em trabalhos
  futuros, é considerar um viés nessa seleção, o que representaria saliência no
  sentido dado por \citeonline{zaller1992simple}: qual questão os agentes estão
  dando atenção, isto é, qual questão está mais acessível na memória deles.}
questão segundo as equações 2.3 e 2.5. Vamos considerar duas variantes de \(p\):
na primeira ele vai ser uma variável global \(0 < p < 1 \), a mesma para todos
os agentes, assim como em \citeonline{martins2009bayesian}; na segunda,
consideramos a possibilidade de que os agentes levem em conta ou não a opinião
do vizinho em uma questão em particular a partir do conhecimento do ponto ideal
do outro agente, de forma que: \(p_{ij}(t) = 1 - |x_i(t) - x_j(t)|\). Isto é,
nesse \(p\) alternativo o agente \(i\) vai considerar a verossimilhança de um
valor declarado pelo seu vizinho \(j\) segundo um \(p\) derivado da média do
posicionamento ideológico de \(j\). Por exemplo, suponha que o agente \(j\) é um
eleitor conservador. O agente \(i\) vai levar
em conta essa informação geral sobre \(j\) para considerar a probabilidade do
agente \(j\) estar falando algo correto em questões particulares.

Os agentes também vão reconsiderar suas opiniões e certezas sobre as questões
segundo uma probabilidade \(\rho\). Do ponto de vista teórico, estamos considerando
a possibilidade de fatores não relacionados à influência social levarem o agente
a mudar seu posicionamento sobre questões \cite{flache2017, lorenz2017modeling}.
Do ponto de vista metodológico, \citeonline{macy2015signal} argumenta que
adicionar ruído contribui para a robustez dos resultados de modelos
teóricos.\todo{discutir mais isso do ruido} 

\section{Parâmetros-Chave}

Os parâmetros-chave para configuração do modelo, cujos valores seguem
\citeonline{martins2008continuous}, \citeonline{deffuant2000mixing} e \citeonline{lorenz2017modeling},  são:
\begin{itemize}
\item A população de \(500 < N < 10000\) agentes;
\item O número de questões \(1 \leq n \leq 10\); 
\item As opiniões \(0.0< o_i< 1.0\) dos agentes
  \begin{itemize}
  \item A opinião dos agentes no tempo \(t = 0\) é retirada de uma distribuição
    uniforme;
  \end{itemize}
\item As incertezas \(0.01 \leq \sigma_i \leq 0.5\);
  \begin{itemize}
  \item A incerteza é, na condição inicial, a mesma para todos os agentes;
  \item Vamos considerar versões em que os agentes atualizam as incertezas e que
    não atualizam.
  \end{itemize}

\item O parâmetro de confiança \(0.1 \leq p \leq 0.99\);
  \begin{itemize}
  \item Vamos considerar variantes em que o \(p\) é global ou em que o \(p\) é
    calculado para cada interação;
  \end{itemize}
  
\item A probabilidade de reconsideração \(0.05 \leq \rho  \leq 0.5\);
  \begin{itemize}
  \item Vamos considerar casos com e sem ruído.
  \end{itemize}
\end{itemize}
























