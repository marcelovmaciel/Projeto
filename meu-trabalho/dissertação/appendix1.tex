A primeira ressalva metodológica em relação aos dados é que não aplicamos os
pesos recomendados pelo \textit{European Social Survey}. Isso significa que,
dado o viés de seleção, as figuras representam o auto-posicionamento dos
respondentes, mas não da população \footnote{Escolhemos a Polônia para a Figura
  3, contudo, por ser o país com a menor variância do peso pós-estratificação.}.
Por não termos aplicado o peso que controla pelo tamanho das populações a Figura
4 não nos permite comparar o auto-posicionamento entre os países\footnote{Mesmo
  que tivéssemos aplicado o peso a comparação entre países usando um
  auto-posicionamento ideológico é complicada dado que a dimensão tem
  significados distintos em diferentes contextos \cite{laver2014measuring}. }.
Além de haver uma variação entre o tamanho das amostras, só representamos,
obviamente, respostas válidas, embora houvesse opção de responder ``Não sei''.



\begin{table}[h]
  \centering
\label{my-label}
\begin{tabular}{|l|r|r|}
\hline
\textbf{Países} & \textbf{N total} & \textbf{Fração  de Respostas  válidas} \\ \hline
Alemanha        & 2919             & 0.93                           \\ \hline
Bélgica         & 1899             & 0.86                           \\ \hline
Dinamarca       & 1506             & 0.93                           \\ \hline
Eslovênia       & 1519             & 0.79                           \\ \hline
Espanha         & 1729             & 0.81                           \\ \hline
Finlândia       & 2000             & 0.95                           \\ \hline
França          & 1503             & 0.94                           \\ \hline
Grécia          & 2566             & 0.77                           \\ \hline
Hungria         & 1685             & 0.83                           \\ \hline
Irlanda         & 2046             & 0.83                           \\ \hline
Israel          & 2499             & 0.92                           \\ \hline
Luxemburgo      & 1552             & 0.77                           \\ \hline
Noruega         & 2036             & 0.98                           \\ \hline
Países Baixos   & 2364             & 0.95                           \\ \hline
Polônia         & 2110             & 0.83                           \\ \hline
Portugal        & 1511             & 0.80                           \\ \hline
Reino Unido     & 2052             & 0.91                           \\ \hline
Suécia          & 1999             & 0.95                           \\ \hline
Suíça           & 2040             & 0.92                           \\ \hline
Áustria         & 2257             & 0.86                           \\ \hline
\end{tabular}
\caption{Número de Entrevistados (N) para 20 países do ESS 2002}
\end{table}


Outra ressalva é que as respostas são discretas (0-10) enquanto a teoria e nosso
modelo supõem pontos ideais num espaço contínuo.

%Uma solução é discretizar o
%\textit{outcome} quando formos validá-lo. Outra é estimar as preferências dos
%indivíduos por meio de suas respostas em outras perguntas do \textit{survey}
%\footnote{Por meio, por exemplo, de técnicas de análise fatorial ou de componentes
  %principais \cite{laver2014measuring}.}.
