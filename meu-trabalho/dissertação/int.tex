Uma característica chave da Democracia é a responsividade do governo às
preferências, crenças e atitudes dos cidadãos
\cite{dahl1973polyarchy,bartels2003democracy}. Nas democracias modernas
(Poliarquias) isso ocorre por meio de vários mecanismos de conexão entres os
cidadãos e seus representantes \cite{dahl1989democracy,
  schumpeter2013capitalism}.

As Democracias estão, desta forma, fundamentadas no nexo entre Opinião Pública e
Governo Representativo. A natureza e origem da Opinião Pública é então
central para a compreensão dos nossos sistemas políticos
\cite{berelson1952democratic}.

Como argumentado por \citeonline{downs1999teoria} a distribuição
da Opinião Pública é de alta relevância para a compreensão do nexo democrático.
Seguindo Downs, o pressuposto central do trabalho é que não só a Opinião Pública
importa, mas, mais especificamente, o seu formato também. Temos, portanto, por
propósito contribuir para a compreensão da emergência da Opinião
Pública.

Todas as interações sociais são ao menos em parte condicionadas pelas crenças e
opiniões dos agentes. É plausível que haja tanto um fundamento
genético-evolucionário para nossas crenças e orientações , quanto uma
fundamentação em mecanismos relacionados a sistemas de herança e aprendizados
sociais \cite{jablonka2014evolution, fowler2008biology, fowler2013defense}.

Um indivíduo ao tomar uma decisão não baseia-se unicamente na informação e
crença averiguadas individualmente, mas também considera as crenças de outros
agentes conectados a ele socialmente e informacionalmente
\cite{gintis2016individuality}. Em política isso significa que as crenças dos
agentes são uma combinação de sua crença ``idiossincrática'' e da combinação de
mensagens/sinais que recebem dos seus pares e da mídia
\cite{barabas2004deliberation,ryan2011social}.

A ``cognição em rede'' \cite{gintis2016individuality} dos agentes e essas duas
fontes de informação, pares e mídia \footnote{O trabalho trata unicamente dos
  pares como fonte de informação. Modelar a influência da mídia é tema para
  trabalhos futuros.}, fornecem assim os dois mecanismos mínimos para
micro-fundamentar abordagens generativas para a Opinião Pública. Por abordagem
generativa entendemos trabalhos guiados pela seguinte pergunta: ``Como pode a
interação local entre agentes autônomos heterogêneos gerar a seguinte
regularidade?'' \cite{epstein2006generative}.

O trabalho, desta forma, tem por \textbf{objetivo geral} explorar, por meio
de um modelo baseado em agentes de influência social, a geração de distribuições
de preferências \footnote{A diferença entre crenças e preferências será
discutida no Capítulo 1.} análogas a distribuições empíricas.

Além desta Introdução, o trabalho é estruturado da seguinte forma: no Capítulo 1
caracterizamos e justificamos o problema do trabalho. Para tal fazemos uma
apresentação do modelo do ator racional e sua aplicação para modelar opinião
pública; no Capítulo 2 fazemos uma revisão de literatura dos modelos canônicos
da área de dinâmicas de opinião e fazemos uma discussão sobre nossa abordagem;
no Capítulo 3 apresentamos o modelo, metodologia de análise e resultados;  por
fim nas Considerações Finais apresentamos limitações do trabalho.








