Uma característica chave da Democracia é a responsividade do governo às
preferências, crenças e atitudes dos cidadãos
\cite{dahl1973polyarchy,bartels2003democracy}. Nas democracias modernas
(Poliarquias) isso ocorre por meio de vários mecanismos de conexão entres os
cidadãos e seus representantes \cite{dahl1989democracy,
  schumpeter2013capitalism}.

Sendo assim, as Poliarquias estão fundamentadas no nexo entre Opinião Pública e
Governo Representativo. Desta forma, a natureza e origem da Opinião Pública é
central para a compreensão dos nossos sistemas políticos
\cite{berelson1952democratic}.

Como canonicamente argumentado por \citeonline{downs1999teoria} a distribuição
da Opinião Pública é de alta relevância para a compreensão do nexo democrático.
Seguindo Downs, o pressuposto central do trabalho é que não só a Opinião Pública
importa, mas, mais especificamente, o seu formato também. Temos, portanto, por
\textbf{objetivo geral} contribuir para a compreensão da emergência da Opinião
Pública.

Todas as interações sociais são ao menos em parte condicionadas pelas crenças e
opiniões dos agentes. Embora seja plausível que haja um fundamento
genético-evolucionário para nossas crenças e orientações
\cite{fowler2008biology, fowler2013defense}, é mais provável que o principal
determinante da variação em  nossas crenças sejam  mecanismos relacionados a 
sistemas de herança e aprendizado sociais \cite{jablonka2014evolution}.

Um indivíduo ao tomar uma decisão não baseia-se unicamente na informação e
crença averiguada individualmente, mas também considera as crenças de outros
agentes conectados a ele socialmente e informacionalmente
\cite{gintis2016individuality}. Em política isso significa que as crenças dos
agentes são uma combinação de sua crença idiossincrática e da combinação de
mensagens/sinais que recebem dos seus pares e da mídia
\cite{barabas2004deliberation,ryan2011social}.


A ``cognição em rede'' dos agentes e essas duas fontes de informação, pares e
mídia \footnote{No trabalho vamos tratar unicamente dos pares como fonte de
  informação. Modelar a influência da mídia é tema para trabalhos futuros.},
fornecem assim os dois mecanismos mínimos para micro-fundamentar abordagens
generativas para a Opinião Pública. Por abordagem generativa entendemos
trabalhos guiados pela seguinte pergunta: ``Como pode a interação a interação
local entre agentes autônomos heterogêneos  gerar a seguinte regularidade?''
\citeonline{epstein2006generative}.

O trabalho, desta forma, tem por \textbf{objetivo específico} explorar, por meio
de um modelo baseado em agentes de influência social, a geração de distribuições
de preferências \footnote{A diferença entre crenças e preferências vai ser
  discutida no Capítulo 1.} análogas a  distribuições empíricas.\footnote{Mais
especificamente, buscamos gerar distribuições de preferências
com formato análogo à distribuição ``estilizada'' na dimensão esquerda-direita
encontrada em \textit{surveys}, tal qual o \textit{European Social Survey}.}




