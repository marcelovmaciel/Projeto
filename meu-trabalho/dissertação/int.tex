Uma característica chave da Democracia é a responsividade do governo às
preferências, crenças e atitudes dos cidadãos
\cite{dahl1989democracy,bartels2003democracy}. Nas democracias modernas isso
ocorre por meio de vários mecanismos de conexão entres os cidadãos e seus
representantes \cite{dahl1989democracy, schumpeter2013capitalism}. As
Democracias estão, desta forma, fundamentadas no nexo entre Opinião Pública e
Governo Representativo. A natureza e origem da Opinião Pública é então central
para a compreensão dos nossos sistemas políticos \cite{berelson1952democratic}.
Como argumentado por \citeonline{downs1999teoria} a distribuição da Opinião
Pública é de alta relevância para a compreensão do nexo democrático na medida em
que orienta o comportamento de políticos e partidos. Seguindo Downs, o
pressuposto central deste trabalho é que não só a Opinião Pública importa, mas,
mais especificamente, o seu formato também. A literatura em ciência política que
busca explicar, por meio de modelos explícitos, o procedimento de surgimento e
dinâmica de padrões de opinião ainda é escassa e desconectada dos trabalhos em
Dinâmicas de Opinião, uma das principais áreas em Simulação Social
\cite{lorenz2017modeling, laver2011party, hauke2017recent}. Há assim espaço para
a contribuição de trabalhos de caráter generativo, isto é, guiados pela seguinte
pergunta: ``Como pode a interação local entre agentes autônomos heterogêneos
gerar a seguinte regularidade?'' \cite{epstein2006generative}.

Pensar em termos generativos nos leva à busca dos microfundamentos das
regularidades de interesse. Todas as interações sociais são ao menos em parte
condicionadas pelas crenças e opiniões dos agentes. É plausível que haja tanto
um fundamento genético-evolucionário para nossas crenças e orientações, quanto
uma fundamentação em mecanismos relacionados a sistemas de herança e
aprendizados sociais \cite{jablonka2014evolution, fowler2008biology,
  fowler2013defense}. Um indivíduo ao tomar uma decisão não baseia-se unicamente
na informação e crença averiguadas individualmente, mas também pode considerar
as crenças de outros agentes conectados a ele socialmente e informacionalmente
\cite{gintis2016individuality}. Em política isso significa que as crenças dos
agentes são uma combinação de sua crença ``idiossincrática'' e da combinação de
mensagens/sinais que recebem dos seus pares e da mídia
\cite{barabas2004deliberation,ryan2011social}. A ``cognição em rede''
\cite{gintis2016individuality} dos agentes e essas duas fontes de informação,
pares e mídia \footnote{O trabalho trata unicamente dos pares como fonte de
  informação. Modelar a influência da mídia é tema para trabalhos futuros.},
fornecem assim os dois mecanismos mínimos para micro-fundamentar abordagens
generativas para a Opinião Pública.

Tendo em vista esses dois mecanismos, este trabalho, tem por \textbf{objetivo
  geral} explorar, por meio de um modelo baseado em agentes de influência
  social, a geração de distribuições de preferências \footnote{A diferença entre
  crenças, preferências e opiniões será analisada nos Capítulo 2 e 3.} análogas
  a distribuições empíricas. O fazemos por meio da adaptação do modelo de
  \citeonline{martins2009bayesian}. Esse modelo em particular é escolhido por
  estar no meio do espectro entre modelos complicados e simples demais do ponto
  de vista cognitivo, além de sua influência bayesiana ser coerente com o
  \textit{framework} do ator racional\footnote{Essas questões são tratada no
  Capítulo 3.}. As modificações foram as seguintes: primeiramente, seguimos a
  recomendação metodológica de \citeonline{macy2015signal} e incluimos ruído na
  dinâmica do modelo; segundo, os posicionamentos dos agentes no espectro
  político são derivados dos posicionamentos deles em várias questões, o que é
  mais consistente com o fato de que alternativas preferidas por agentes em
  contextos de massa são derivadas das atitudes deles em diferentes
  questões\footnote{Como discutido nos Capítulos 2 e 4.}; ademais, adicionamos
  agentes intransigentes à população, o que é uma variante, possibilitada pelo
  modelo de \citeonline{martins2009bayesian}, da verificação do papel de
  extremistas\footnote{ Os extremistas são inseridos nos modelos com o intuito
  de investigar quais propriedades de agentes e de suas interações levam ao
  consenso, à populações isoladas ou polarizadas. Isso é tema constante de
  debate e reflexão política, em decorrência de sua conexão com a
  estabilidade/instabilidade de governos e seus efeitos societais. Não por acaso
  é também um dos problemas de pesquisa centrais para a área de Dinâmicas de
  Opinião \cite{fiorina2005culture, pulick2016, bramson2016disambiguation}.} na
  dinâmica de opinião \cite{deffuant2002can, flache2017}. O modelo foi analisado
  por meio da combinação de análise de sensibilidade, histogramas e séries
  temporais de medidas do sistema e gráficos de dispersão, com o intuito de
  verificar em que medida essas modificações levam a mudanças de comportamento
  em contraste ao modelo original e quais os efeitos dos atributos dos agentes e
  parâmetros do modelo sobre a opinião populacional, mais precisamente sobre a
  dispersão e cobertura das opiniões\footnote{Ambas medidas de polarização
  abordadas no Capítulo 4.}.

Além desta Introdução, o trabalho é estruturado da seguinte forma: no Capítulo 2
caracterizamos e justificamos o problema do trabalho. Para tal apresentamos os
fundamentos do modelo do ator racional, como é adaptado para o contexto político
por meio dos modelos ``espaciais'' de política e a aplicação desses modelos para
representar a opinião pública. Examinamos a relação entre preferências e
atitudes e qual o impacto disso para pensar os agentes em contextos
macro-políticos; no Capítulo 3 definimos a área de Dinâmicas de Opinião (OD),
discorremos sobre o que são Modelos Baseados em Agentes e os elementos
constituintes de um modelo de OD, fazemos uma revisão de literatura dos modelos
canônicos da área e fazemos uma discussão sobre onde nossa abordagem está
localizada dentre as várias formas de se modelar as atitudes de agentes; no
Capítulo 4 apresentamos as entidades e processos do modelo, quais seus
parâmetros, especificamos como é realizada a inicialização e o procedimento do
modelo, em seguida discutimos a metodologia de análise e os resultados; por fim
na Conclusão recapitulamos o que foi feito e tratamos das limitações do
trabalho.



