\documentclass{article}

\usepackage[brazilian]{babel}
\usepackage[utf8]{inputenc}
\usepackage[T1]{fontenc}
\usepackage{tcolorbox}
\usepackage{amsmath}
\usepackage{amsfonts}
\usepackage{lmodern}


% bib management

\usepackage[style = abnt, backend = biber]{biblatex}
%\addbibresource{refs.bib}

% This allows us to add our colors
\usepackage{xcolor}

%just input either solarized light or dark to change between the colors
%\input{solarizeddark.tex}
\input{solarizedlight.tex}
\input{styling.tex}


\title{\textcolor{red}{Comentários da Banca}}
\author{Marcelo Veloso Maciel } \date{}

\begin{document}

% Global Page color base03
\pagecolor{base03}
% Global Text color base1
\color{base1}
\maketitle


\section*{Comentários de Masayuki}
\begin{itemize}
\item Ajeitar a equação errada;
\item Ver melhor o que a rede é no modelo;
\item Ver Rednick em relação ao modelo da maioria\footnote{Que eu pensava que
    fosse de Galam...}
\item Unidimensional o que? Explicar isso (esqueci...)
\item Conclusao ta porca dshrsieher 
\end{itemize}

\section*{Comentários de Carlos Brito}
\begin{itemize}
\item Papel do tempo; qual o tempo equivalente na vida real?
\item Ver validação x confirmação (estatisticamente)
\item Explicar melhor as figuras;
\item Rever a nóia com canônico;
\item Explicar melhor teoria formal (ou cortar...)
\item Justificar o uso de ABMs melhor
\item Justificar o valor dos parâmetros
\item Explicar melhor a parte de não uso de utilidade...
\item Justificar o uso de Julia;
\item Discutir melhor ruído
\end{itemize}


\section*{Comentários de Camilo}
\begin{itemize}
\item Ver Monte Carlo Steps
\item Redes : entender melhor o que são as redes no trabalho
\item Simular até o estado estacionário?

\end{itemize}

\section*{Sumarizando}
\textbf{Ver os comments  e ajeitar um por um...}
%\printbibliography

\end{document}

%%% Local Variables:
%%% mode: latex
%%% TeX-master: t
%%% End:
