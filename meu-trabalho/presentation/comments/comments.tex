\documentclass{article}

\usepackage[brazilian]{babel}
\usepackage[utf8]{inputenc}
\usepackage[T1]{fontenc}
\usepackage{tcolorbox}
\usepackage{amsmath}
\usepackage{amsfonts}
\usepackage{lmodern}
\usepackage{dirtytalk}

% bib management

\usepackage[style = abnt, backend = biber]{biblatex}
%\addbibresource{refs.bib}

% This allows us to add our colors
\usepackage{xcolor}

%just input either solarized light or dark to change between the colors
%\input{solarizeddark.tex}
\input{solarizedlight.tex}
\input{styling.tex}


\title{\textcolor{red}{Comentários da Banca}}
\author{Marcelo Veloso Maciel } \date{}

\begin{document}

% Global Page color base03
\pagecolor{base03}
% Global Text color base1
\color{base1}
\maketitle


\section*{Comentários de Masayuki}
\begin{itemize}
\item Ajeitar a equação errada;  \textcolor{red}{DONE}
\item Ver melhor o que a rede é no modelo;
\item Ver Rednick em relação ao modelo da maioria\footnote{Que eu pensava que
    fosse de Galam...}
\item Unidimensional o que? Explicar isso (esqueci...)  \textcolor{red}{DONE}
\item Conclusao ta porca dshrsieher \textcolor{red}{DONE}
\end{itemize}

\section*{Comentários de Carlos Brito}
\begin{itemize}
\item Papel do tempo; qual o tempo equivalente na vida real?
\item Ver validação x confirmação (estatisticamente)
\item Explicar melhor as figuras;
\item Rever a nóia com canônico; \textcolor{red}{DONE}
\item Explicar melhor teoria formal (ou cortar...)
\item Justificar o uso de ABMs melhor
\item Justificar o valor dos parâmetros  \textcolor{red}{DONE}
\item Explicar melhor a parte de não uso de utilidade...
\item Justificar o uso de Julia; \textcolor{red}{DONE}
\item Discutir melhor ruído
  \item Explicar o que é \textit{benchmark}
\end{itemize}


\section*{Comentários de Camilo}
\begin{itemize}
\item Ver Monte Carlo Steps
\item Redes : entender melhor o que são as redes no trabalho
\item Simular até o estado estacionário?
\end{itemize}

\section*{Comentários de André}
\begin{itemize}
\item introdução de \(p\) como viés de confirmação não é a única interpretação
\item Papel do número de questões? Teorema central do limite... quanto mais
  questões mais central vai ser o resultado;
\item explicar melhor a escolha dos elementos do modelo \(\rightarrow\) acertar a
  descrição do modelo \(\rightarrow\) como os elementos entram no modelo e porque entra
  dessa forma;
\item Unidimensional? \textcolor{red}{DONE}
\end{itemize}

Quote dele:
\say{Em primeiro lugar, entendo a lógica dos assuntos que gerariam a opinião final sobre um tema. Mas experimentos em psicologia mostram que a lógica costuma ser inversa. As pessoas costumam escolher a conclusão e, a partir dela, selecionar os argumentos que a apoiam. Claro que estas conclusões ainda vão se juntar em um tema maior, nesse ponto, a descrição de pegar assuntos com decisões aleatórias me parece menos ruim. Ainda assim, ao ter muitos assuntos, você acaba que ninguém teria uma opinião central nos extremos o que me parece estranho. Mas poderia ser um estado inicial antes de qualquer interação social. A questão da conclusão vir primeiro é um trabalho já antigo do Jervis: Jervis R. Perception and Misperception in International Politics. Princeton: Princeton University Press (1976).

  Quanto à descrição do modelo, pode valer a pena utilizar uma das sugestões de descrição completa. Há protocolos para isso, por exemplo, no JASSS. Veja por exemplo

\url{http://jasss.soc.surrey.ac.uk/9/1/15.html}
ou
\url{http://jasss.soc.surrey.ac.uk/11/2/3.html}
Em especial, deixe claro o como essas opiniões sobre pequenos assuntos vão entrar (ou não) na simulação.}

\section*{Sumarizando}
\textbf{Ver os comments  e ajeitar um por um...}
%\printbibliography

\end{document}

%%% Local Variables:
%%% mode: latex
%%% TeX-master: t
%%% End:
