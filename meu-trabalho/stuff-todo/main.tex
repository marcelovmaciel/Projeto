% While LaTeX prints out all nice and pretty, most of the time we stare at the PDF on our computer screen
% Therefore, we should have a color scheme and fonts that are easy on the eye on a screen.
% So I took an idea from my programming IDE's and here is Solarized Dark LaTeX

% Created by Christina Lee, christina.lee@gmail.com

\documentclass{article}
\usepackage[utf8]{inputenc}

% This allows us to add our colors
\usepackage{xcolor}

%just input either solarized light or dark to change between the colors
%\input{solarizeddark.tex}
\input{solarizedlight.tex}

\input{styling.tex}


\title{\textcolor{red}{A fazer da dissertação}}
\author{Marcelo Veloso Maciel}
\date{}

\begin{document}

% Global Page color base03
\pagecolor{base03}
% Global Text color base1
\color{base1}

\maketitle


\section*{Prioridade}
\begin{itemize}
\item Ajeitar o que falta do capítulo 1
\item Ver se o capítulo 3 segue razoavelmente o ODD
\item Comentar as funções e Modularizar a geração dos dados e a geração de gráficos;
\end{itemize}

\section*{A fazer geral}
\begin{itemize}
\item Ler sobre analise de ABMs:
  \begin{itemize}
  \item Capítulo de analise de Laver
  \item (ver focus modelling1)
  \end{itemize}
  \item ver os comments da presentation e ajeitar!!
  \item Ajeitar a apresentação do chap3!!:
    \begin{itemize}
    \item ver os comments de Andre.
    \end{itemize}
\end{itemize}





\end{document}

%%% Local Variables:
%%% mode: latex
%%% TeX-master: t
%%% End:
