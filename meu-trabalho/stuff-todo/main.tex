% While LaTeX prints out all nice and pretty, most of the time we stare at the PDF on our computer screen
% Therefore, we should have a color scheme and fonts that are easy on the eye on a screen.
% So I took an idea from my programming IDE's and here is Solarized Dark LaTeX

% Created by Christina Lee, christina.lee@gmail.com

\documentclass{article}
\usepackage[utf8]{inputenc}

% This allows us to add our colors
\usepackage{xcolor}

%just input either solarized light or dark to change between the colors
%\input{solarizeddark.tex}
\input{solarizeddark.tex}

\input{styling.tex}


\title{\textcolor{red}{A fazer da dissertação}}
\author{Marcelo Veloso Maciel}
\date{}

\begin{document}

% Global Page color base03
\pagecolor{base03}
% Global Text color base1
\color{base1}

\maketitle

\section*{Prioridades}

\begin{itemize}
\item {\Large Ver o diss-struct}
\end{itemize}

\section*{A fazer geral}

\begin{itemize}

\item {\Large Geral - ver os comentarios de Andre. tem uns que já eram
    esperados. Tem outros que é noobice minha mesmo. E tem uns que talvez seja
    melhor eu reestruturar}
\item {\Large C2 - ajeitar a revisão; discutir melhor o resultado base de cada
    modelo e colocar as figuras para cada}
\item {\Large Refs - ajeitar as citações e referencias (ta meio inconsistente
    por culpa do google)}.

\end{itemize}

\section*{Leituras para a dissertação}

\begin{itemize}
\item {\Large \textbf{A ementa de Nara. Sério, preciso da base em opinião
      publica presse trabalho ser decente.};}
\item Sobcowiz
\item sznajd 2014;
\item urbig 2008; 
\item lorenz 2007;
\item jagger 2005;
\end{itemize}

\section*{Pra apresentação}

\begin{itemize}
\item Falar do objeto
\item Falar de teoria formal e dos pressupostos
\item Falar da definição de OD
\item Falar dos modelos, da area e dos extras
\item Falar da contribuição de OD e do uso do BC dado que a variavel é continua
\item Falar do framework de andre como alternativa
\item Falar da forma funcional não obvia $\land$ Falar do dilema dos niveis e da
  tratabilidade
\item Da não obviedade puxar o debate sobre a comparação entre modelos
\item Apresentar melhor os modelos
\item Falar do plano de testar mais rigorosamente no chap 4 
\item Prospects:
  \begin{itemize}
  \item  O problema da Escala  - N e tempo.
  \item A necessidade do teste comparativo de modelos
  \item A necessidade de frameworks para a área. (IAD como exemplo $\rightarrow$ niveis,
    elementos relevantes.)
  \end{itemize}
\end{itemize}


\end{document}
