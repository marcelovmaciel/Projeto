% While LaTeX prints out all nice and pretty, most of the time we stare at the PDF on our computer screen
% Therefore, we should have a color scheme and fonts that are easy on the eye on a screen.
% So I took an idea from my programming IDE's and here is Solarized Dark LaTeX

% Created by Christina Lee, christina.lee@gmail.com

\documentclass{article}
\usepackage[utf8]{inputenc}

% This allows us to add our colors
\usepackage{xcolor}

%just input either solarized light or dark to change between the colors
%% Background Tones
\definecolor{base03}{HTML}{002b36} % for solarized dark
\definecolor{base02}{HTML}{073642} 
\definecolor{base2}{HTML}{eee8d5} % for solarized light
\definecolor{base3}{RGB}{253,246,227}

% Content Tones
\definecolor{base01}{HTML}{586e75}
\definecolor{base00}{HTML}{657b83}
\definecolor{base0}{HTML}{839496}
\definecolor{base1}{HTML}{93a1a1}

% Accent Colors
\definecolor{cyan}{HTML}{2aa198}
\definecolor{violet}{HTML}{6c71c4}
\definecolor{yellow}{HTML}{b58900}
\definecolor{orange}{HTML}{cb4b16}
\definecolor{red}{RGB}{220,50,47}
\definecolor{magenta}{HTML}{d33682}
\definecolor{blue}{HTML}{268bd2}
\definecolor{cyan}{HTML}{2aa198}
\definecolor{green}{HTML}{859900}
% Background Tones
\definecolor{base03}{HTML}{002b36} % for solarized dark
\definecolor{base02}{HTML}{073642} 
\definecolor{base2}{HTML}{eee8d5} % for solarized light
\definecolor{base3}{RGB}{253,246,227}

% Content Tones
\definecolor{base01}{HTML}{586e75}
\definecolor{base00}{HTML}{657b83}
\definecolor{base0}{HTML}{839496}
\definecolor{base1}{HTML}{93a1a1}

% Accent Colors
\definecolor{cyan}{HTML}{2aa198}
\definecolor{violet}{HTML}{6c71c4}
\definecolor{yellow}{HTML}{b58900}
\definecolor{orange}{HTML}{cb4b16}
\definecolor{red}{RGB}{220,50,47}
\definecolor{magenta}{HTML}{d33682}
\definecolor{blue}{HTML}{268bd2}
\definecolor{cyan}{HTML}{2aa198}
\definecolor{green}{HTML}{859900}

\usepackage{tcolorbox}
\tcbset
{every box/.style={colback=base03,colframe=blue,boxrule=2mm,bottom=-7pt}}

% We will use this to alter commands and environments, making equations and tables different colors
\usepackage{etoolbox}
% All equation environments are base2
\AtBeginEnvironment{equation}{\color{base2}}
\AtBeginEnvironment{equation*}{\color{base2}}
\AtBeginEnvironment{math}{\color{violet}}
\AtBeginEnvironment{align}{\color{base2}}
\AtBeginEnvironment{tabular}{\color{violet}}
\usepackage{cancel}
\renewcommand{\CancelColor}{\color{red}}

% A nice sans- serif font
\usepackage[default,osfigures,scale=0.95]{opensans}

% Un-italicized the equations and makes them easier to read. Check out
% http://www.biwako.shiga-u.ac.jp/sensei/kumazawa/tex/newtx.html
% for more options
\usepackage{amsmath}
\usepackage{cmbright}

% Doesn't seem to be working in Share LaTeX
% Gives you control over footer, header, etc. so we can have a highlighter page number
% We also have the section title up on top
\usepackage{fancyhdr}
\pagestyle{fancy}
\cfoot{\textcolor{violet}{\thepage} }
% section title on the left side of the page
\fancyhead[L]{\textcolor{violet}{\slshape \leftmark}}
\fancyhead[R]{}

% Allows us to style section headings
\usepackage{sectsty}
\sectionfont{\color{yellow}}
\subsectionfont{\color{orange}}
\subsubsectionfont{\color{green}}

%Control over figure captions.  I bolded them as well
\usepackage[font={color=cyan,bf}]{caption}



\title{\textcolor{red}{A fazer da dissertação}}
\author{Marcelo Veloso Maciel}
\date{}

\begin{document}

% Global Page color base03
\pagecolor{base03}
% Global Text color base1
\color{base1}



\maketitle

\section*{Prioridades}

\begin{itemize}
  
\item {\Large C2 - ESS  };  
\item {\Large C3 - implementar modelos BC e modelo de Andre }
  
\end{itemize}

\section*{A fazer}

\begin{itemize}

\item {\Large Geral - ver os comentarios de Andre. tem uns que já eram
    esperados. Tem outros que é noobice minha mesmo. E tem uns que talvez seja
    melhor eu reestruturar\footnote{\textcolor{base01}{Ver a seção ``A
        considerar''}}}
\item {\Large Int - usar a int de acemoglu como inspiração p enrolação da minha
    int (deixar isso pra depois da quali)}
\item {\Large C1  - fundamentar melhor a parte de teoria formal}:
  \begin{itemize}

  \item Buscar em Barber a definição de política
  \item Buscar o artigo ``what is political theory''
  \item Ler o artigo de historia da Social Choice no handbook;
  \item Ler , por cima, o  final de Elster 2015;
  \item Ler, por cima,  o artigo de austen-smith.
  \item Ler a entrada na Stanford encyclopedia sobre preferencias.
    
  \end{itemize}

\item {\Large C1 - ajeitar a revisão; discutir melhor o resultado base de cada
    modelo e colocar as figuras para cada}
\item {\Large C1 - sacar aldrich 1993 em morton. altamente relevante pra minha
    argumentação do fim do C1};
\item {\Large  C1- rever meu argumento que a função de utilidade em politica
    difere da de economia. em economia tbm tem bliss point!! };
\item {\Large C2 - Discutir o pressuposto da normalidade, tomar Laver como
    exemplo. Procurar uns artigos que façam o mesmo. Mostrar que não é assim
    puxando de Lorenz. Puxar a discussão de modelos de OD segundo a
    clusterização (Lorenz et al 2017 etc) e dar o gancho ao 3};
\item {\Large C3 - fundamentar melhor o argumento da separação entre crenças e
    preferencias}
\item {\Large C3 - fundamentar melhor o debate sobre simplicidade, niveis,
    cognição usando jagger 2017, frontiers, De Marchi, uskali maki e binmore!!
    Ta mal argumentado \textbf{mesmo} }:
  \begin{itemize}
  \item Como assim niveis? Você colocar um agente no modelo não define a priori
    o seu foco com o modelo. Num extremo nos temos modelos de arquiteturas
    cognitivas, cujo foco é estudar no nivel individual a cognição. No outro,
    temos modelos em que os agentes na real sao mero ``placeholders'' para
    efeitos estruturais. Dentre esses extremos temos uma gama de abordagens;
  \item Isso significa que existem diferentes formas de modelar agentes, não
    existe um ``form'' universal que seja a aplicado para todo fenômeno. O nivel
    de ``complicação'' que queremos dar aos nossos agentes depende do nosso
    objetivo.
  \item Não é dificil pensar em modelos mais propriamente sociologicos, a la
    durkheim, que estão preocupados com situações de ação, normas, estruturas,
    isto é, fatos sociais, que são supervenientes tanto à cognição dos agentes,
    quanto aos mecanismos de psicologia social. Quando a psicologia importa para
    estudar fatos sociais é comum que cientistas sociais façam uso de uma ``folk
    psychology'' (maki aqui).
  \item Em dinamicas de opinião é comum querermos modelar mecanismos típicos de
    ``psicologia social'', isto é, mecanismos que afetam atributos dos agentes
    pela via cognitiva, mas que só fazem sentido por serem sociais,
    interacionistas (procurar algo sobre psicologia social para fundamentar)
  \item O BC modela um efeito ou mecanismo (ver melhor como chamar) desse:
    pessoas ao interagirem se aproximam, ficam neutras, ou ate se repelem, etc.
    Isso é algo tipicamente estudado por psicologia social.
  \item O BC atribui uma forma funcional para essa relação, mas outras são
    possiveis. Não existe uma relação direta entre a descoberta empírica do
    fenomeno e a forma funcional que usamos para tentar representa-lo. Isto é,
    ele atribui uma heurística para os agentes que busca capturar esse fenomeno
    típico em psicologia social. Essa heurística contudo foi atribuída sem
    inspirar-se em algum quadro formal anterior, e que tenha sido testado no lab
    etc. É nesse sentido que ela é arbitrária. Não buscar um framework
    integrativo tem levado OD à proliferação de vários modelos que não dialogam,
    e cuja heuristica representa algum fenomeno em psicologia social, mas foi
    arbritariamente atribuida aos agentes pelo pesquisador.
  \item O framework proposto por andre captura esse fenomeno em psicologia
    social, mas com um quadro inspirado em Teoria da Decisão.  O modelo dele é
    mais complicado, pressupõe mais dos agentes (é mais ``cognitivamente
    denso''), mas tem maior fundamentação empírica, dado que muitos estudos
    mostram que nós seres humanos não somos tão distantes assim dos agentes da
    teoria da decisão e da utilidade, e o quadro formal dessas teorias é na
    verdade uma boa aproximação muitas vezes. Além disso o modelo de andre,
    continuo, endogeniza o limiar de confiança.
  \item Inspirar-se em teoria da decisão, e apresentar uma estrategia para como
    faze-lo dá um papel fundacional ao ``framework'' de andre.
  \end{itemize}
\item {\Large Refs - ajeitar as citações e referencias (ta meio inconsistente
    por culpa do google)}.

\end{itemize}

\section*{Leituras para a dissertação}

\begin{itemize}
\item {\Large \textbf{A ementa de Nara. Sério, preciso da base em opinião
      publica presse trabalho ser decente.};}
\item Sobcowiz
\item sznajd 2014;
\item urbig 2008; 
\item lorenz 2007;
\item jagger 2005;
\end{itemize}


\section*{A considerar}

Mandar um email para andre antes de fazer essas modificações, mas hoje \today
\hspace{0.1cm} me parecem razoáveis.

\begin{itemize}
\item Reestruturar o C1. Ok, a divisao da area segundo tres criterios (sistema
  alvo, modelos usados, e esfera de pesquisadores) faz total sentido, mas mesmo
  que as pessoas não se reconheçam como da mesma área é interessante eu
  apresentar os trabalhos Bayesianos (acemoglu e afins) feitos na economia e os trabalhos de
  modelagem em opinião publica (os neurocognitivos, os bayesianos, e os mais a
  la psicologia social). Posso deixar claro que tou apresentando para que seja
  completo, mas que esses trabalhos não fulfill a terceira condição.
  Apresenta-los aqui significa que a discussao do C3 fica mais clara e
  fundamentada.
\item Outra mudança é jogar a ultima seção sobre teoria formal para o C2.
  Estrutura plausivel:
  \begin{itemize}
  \item supondo que a primeira modificação foi feita, puxar das abordagens em
    opinião publica para uma apresentação da teoria formal e da abordagem
    espacial. DAI eu entro no debate sobre normalidade; apresento os dados do
    ESS, e falo da contribuição da area de dinamicas de opinião para área.
  \item na definição de teoria politica formal, primeiro dar uma definição
    generalista com base em epistemologia (teoria política formal: definição
    semantica de teoria + definição de política (puxar ordeshook e
    oppenheimer)); depois puxar as escolas de escolha racional + depois puxar
    sistemas multiagentes, para então chegar no debate sobre preferencias,
    preferencias espaciais e normalidade. 
  \end{itemize}
\end{itemize}


\section*{Pra apresentação}

\begin{itemize}
\item Falar do objeto
\item Falar da definição da area
\item Falar dos modelos, da area e dos extras
\item Falar de teoria formal e dos pressupostos
\item Falar da contribuição de OD e do uso do BC dado que a variavel é continua
\item Falar do framework de andre como alternativa
\item Falar da forma funcional não obvia $\land$ Falar do dilema dos niveis e da
  tratabilidade
\item Da não obviedade puxar o debate sobre a comparação entre modelos
\item Apresentar melhor os modelos
\item Falar do plano de testar mais rigorosamente no chap 4 
\item Prospects:
  \begin{itemize}
  \item  O problema da Escala  - N e tempo.
  \item A necessidade do teste comparativo de modelos
  \item A necessidade de frameworks para a área. (IAD como exemplo $\rightarrow$ niveis,
    elementos relevantes.)
  \end{itemize}


\end{itemize}


\end{document}
