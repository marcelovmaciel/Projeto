\documentclass{article}
\usepackage[utf8]{inputenc}

% This allows us to add our colors
\usepackage{xcolor}
\usepackage{tcolorbox}
%just input either solarized light or dark to change between the colors
%% Background Tones
\definecolor{base03}{HTML}{002b36} % for solarized dark
\definecolor{base02}{HTML}{073642} 
\definecolor{base2}{HTML}{eee8d5} % for solarized light
\definecolor{base3}{RGB}{253,246,227}

% Content Tones
\definecolor{base01}{HTML}{586e75}
\definecolor{base00}{HTML}{657b83}
\definecolor{base0}{HTML}{839496}
\definecolor{base1}{HTML}{93a1a1}

% Accent Colors
\definecolor{cyan}{HTML}{2aa198}
\definecolor{violet}{HTML}{6c71c4}
\definecolor{yellow}{HTML}{b58900}
\definecolor{orange}{HTML}{cb4b16}
\definecolor{red}{RGB}{220,50,47}
\definecolor{magenta}{HTML}{d33682}
\definecolor{blue}{HTML}{268bd2}
\definecolor{cyan}{HTML}{2aa198}
\definecolor{green}{HTML}{859900}
% Background Tones
\definecolor{base03}{HTML}{002b36} % for solarized dark
\definecolor{base02}{HTML}{073642} 
\definecolor{base2}{HTML}{eee8d5} % for solarized light
\definecolor{base3}{RGB}{253,246,227}

% Content Tones
\definecolor{base01}{HTML}{586e75}
\definecolor{base00}{HTML}{657b83}
\definecolor{base0}{HTML}{839496}
\definecolor{base1}{HTML}{93a1a1}

% Accent Colors
\definecolor{cyan}{HTML}{2aa198}
\definecolor{violet}{HTML}{6c71c4}
\definecolor{yellow}{HTML}{b58900}
\definecolor{orange}{HTML}{cb4b16}
\definecolor{red}{RGB}{220,50,47}
\definecolor{magenta}{HTML}{d33682}
\definecolor{blue}{HTML}{268bd2}
\definecolor{cyan}{HTML}{2aa198}
\definecolor{green}{HTML}{859900}
\usepackage{tcolorbox}
\tcbset
{every box/.style={colback=base03,colframe=blue,boxrule=2mm,bottom=-7pt}}

% We will use this to alter commands and environments, making equations and tables different colors
\usepackage{etoolbox}
% All equation environments are base2
\AtBeginEnvironment{equation}{\color{base2}}
\AtBeginEnvironment{equation*}{\color{base2}}
\AtBeginEnvironment{math}{\color{violet}}
\AtBeginEnvironment{align}{\color{base2}}
\AtBeginEnvironment{tabular}{\color{violet}}
\usepackage{cancel}
\renewcommand{\CancelColor}{\color{red}}

% A nice sans- serif font
\usepackage[default,osfigures,scale=0.95]{opensans}

% Un-italicized the equations and makes them easier to read. Check out
% http://www.biwako.shiga-u.ac.jp/sensei/kumazawa/tex/newtx.html
% for more options
\usepackage{amsmath}
\usepackage{cmbright}

% Doesn't seem to be working in Share LaTeX
% Gives you control over footer, header, etc. so we can have a highlighter page number
% We also have the section title up on top
\usepackage{fancyhdr}
\pagestyle{fancy}
\cfoot{\textcolor{violet}{\thepage} }
% section title on the left side of the page
\fancyhead[L]{\textcolor{violet}{\slshape \leftmark}}
\fancyhead[R]{}

% Allows us to style section headings
\usepackage{sectsty}
\sectionfont{\color{yellow}}
\subsectionfont{\color{orange}}
\subsubsectionfont{\color{green}}

%Control over figure captions.  I bolded them as well
\usepackage[font={color=cyan,bf}]{caption}



\title{\textcolor{red}{Estrutura da Dissertação}}
\author{Marcelo Veloso Maciel}
\date{}

\begin{document}

% Global Page color base03
\pagecolor{base03}
% Global Text color base1
\color{base1}
\maketitle

\newtcolorbox{mybox}[1]{colback=base03,
  colframe=base00,fonttitle=\bfseries, title=#1}

\begin{abstract}
  Estrutura da Dissertação a partir da conversa com Fernando
\end{abstract}

\section*{Introdução}

Aqui apresento o tema de maneira generica (opinião em geral, opinião publica,
nexo democratico, distribuição de preferencias) e apresento o objetivo do
trabalho :

\begin{mybox}{Objetivo}
  \textcolor{base1}{\textbf{O trabalho tem por objetivo explorar, por meio de
      simulação, a geração de distribuições de preferências da população.}}
  \begin{mybox}{Objetivos Específicos}
    \textcolor{base1}{\begin{itemize}
    \item Simular, por meio de modelos baseados em agentes, a geração de
      distribuições de preferências, formalizadas como pontos ideais.
    \item Dentro do quadro estabelecido por Andre (2012), e a partir do modelo
      de 2009 simular uns variantes (mudança aleatória + topologias distintas)
    \item Validar esses variantes: ver qual melhor se aproxima da distribuição
      empírica;
    \end{itemize}}
    \vspace{0.2cm}
  \end{mybox}
\end{mybox}

\section*{Capítulo 1 - Distribuição de preferencias e Teoria Espacial }

Discutir teoria espacial e a importancia de estudar distribuições

\section*{Capítulo 2 - Revisão de OD }

Discutir modelos canônicos, discutir a relação entre preferencias e opiniões,
discutir porque o framework de andre.


\section*{Capítulo 3 - Apresentação dos Modelos}

Modelo + Resultados Parciais


\section*{Considerações Finais}

Limitações, o que falta, cronograma.



\end{document}