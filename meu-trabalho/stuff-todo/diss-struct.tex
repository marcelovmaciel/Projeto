\documentclass{article}
\usepackage[utf8]{inputenc}

% This allows us to add our colors
\usepackage{xcolor}
\usepackage{tcolorbox}
%just input either solarized light or dark to change between the colors
%% Background Tones
\definecolor{base03}{HTML}{002b36} % for solarized dark
\definecolor{base02}{HTML}{073642} 
\definecolor{base2}{HTML}{eee8d5} % for solarized light
\definecolor{base3}{RGB}{253,246,227}

% Content Tones
\definecolor{base01}{HTML}{586e75}
\definecolor{base00}{HTML}{657b83}
\definecolor{base0}{HTML}{839496}
\definecolor{base1}{HTML}{93a1a1}

% Accent Colors
\definecolor{cyan}{HTML}{2aa198}
\definecolor{violet}{HTML}{6c71c4}
\definecolor{yellow}{HTML}{b58900}
\definecolor{orange}{HTML}{cb4b16}
\definecolor{red}{RGB}{220,50,47}
\definecolor{magenta}{HTML}{d33682}
\definecolor{blue}{HTML}{268bd2}
\definecolor{cyan}{HTML}{2aa198}
\definecolor{green}{HTML}{859900}
% Background Tones
\definecolor{base3}{HTML}{002b36} % for solarized dark
\definecolor{base2}{HTML}{073642} 
\definecolor{base02}{HTML}{eee8d5} % for solarized light
\definecolor{base03}{RGB}{253,246,227}

% Content Tones
\definecolor{base1}{HTML}{586e75}
\definecolor{base0}{HTML}{657b83}
\definecolor{base00}{HTML}{839496}
\definecolor{base01}{HTML}{93a1a1}

% Accent Colors
\definecolor{cyan}{HTML}{2aa198}
\definecolor{violet}{HTML}{6c71c4}
\definecolor{yellow}{HTML}{b58900}
\definecolor{orange}{HTML}{cb4b16}
\definecolor{red}{RGB}{220,50,47}
\definecolor{magenta}{HTML}{d33682}
\definecolor{blue}{HTML}{268bd2}
\definecolor{cyan}{HTML}{2aa198}
\definecolor{green}{HTML}{859900}
\usepackage{tcolorbox}
\tcbset
{every box/.style={colback=base03,colframe=blue,boxrule=2mm,bottom=-7pt}}

% We will use this to alter commands and environments, making equations and tables different colors
\usepackage{etoolbox}
% All equation environments are base2
\AtBeginEnvironment{equation}{\color{base2}}
\AtBeginEnvironment{equation*}{\color{base2}}
\AtBeginEnvironment{math}{\color{violet}}
\AtBeginEnvironment{align}{\color{base2}}
\AtBeginEnvironment{tabular}{\color{violet}}
\usepackage{cancel}
\renewcommand{\CancelColor}{\color{red}}

% A nice sans- serif font
\usepackage[default,osfigures,scale=0.95]{opensans}

% Un-italicized the equations and makes them easier to read. Check out
% http://www.biwako.shiga-u.ac.jp/sensei/kumazawa/tex/newtx.html
% for more options
\usepackage{amsmath}
\usepackage{cmbright}

% Doesn't seem to be working in Share LaTeX
% Gives you control over footer, header, etc. so we can have a highlighter page number
% We also have the section title up on top
\usepackage{fancyhdr}
\pagestyle{fancy}
\cfoot{\textcolor{violet}{\thepage} }
% section title on the left side of the page
\fancyhead[L]{\textcolor{violet}{\slshape \leftmark}}
\fancyhead[R]{}

% Allows us to style section headings
\usepackage{sectsty}
\sectionfont{\color{yellow}}
\subsectionfont{\color{orange}}
\subsubsectionfont{\color{green}}

%Control over figure captions.  I bolded them as well
\usepackage[font={color=cyan,bf}]{caption}



\title{\textcolor{red}{Estrutura da Dissertação}}
\author{Marcelo Veloso Maciel}
\date{}

\begin{document}

% Global Page color base03
\pagecolor{base03}
% Global Text color base1
\color{base1}
\maketitle

\newtcolorbox{mybox}[1]{colback=base03,
  colframe=base00,fonttitle=\bfseries, title=#1}

\begin{abstract}
  Estrutura da Dissertação a partir da conversa com Fernando
\end{abstract}

\section*{Introdução}

Aqui apresento o tema de maneira generica (opinião em geral, opinião publica,
nexo democratico, distribuição de preferencias) e apresento o objetivo do
trabalho:

\begin{mybox}{Objetivo}
  \textcolor{base1}{\textbf{O trabalho tem por objetivo explorar, por meio de
      simulação, a geração de distribuições de preferências da população.}}
  \begin{mybox}{Objetivos Específicos}
    \textcolor{base1}{\begin{itemize}
    \item Simular, por meio de modelos baseados em agentes,alyzed, Visualized, Shared a geração de
      distribuições de preferências, formalizadas como pontos ideais.
    \item Dentro do quadro estabelecido por Andreordeshook1990emerging (2012), e a partir do modelo
      de 2009 simular uns variantes (mudança aleatória + topologias distintas)
    \item Validar esses variantes: ver qual melhor se aproxima da distribuição
      empírica;
    \end{itemize}}
    \vspace{0.2cm}
  \end{mybox}
\end{mybox}



\begin{itemize}
\item {\Large \textcolor{red}{DONE}}
\end{itemize}


\section*{Capítulo 1 - Distribuição de preferencias e Teoria Espacial }
\begin{itemize}
\item Discuto que vou focar na abordagem da Teoria Politica Formal para a
  questão.\textcolor{red}{DONE};
\item discuto o que estou considerando como \textcolor{red}{DONE}:
  \begin{itemize}
  \item Teoria (conjunto de modelos+ hipoteses teoricas - clarke $\land$ primo);
  \item Politica (collective choice and action - Oppenheimer); 
  \item e Formal (uso de logica formal, matematica e simulações- Morton).
\end{itemize}
\item Discuto um pouco de historia da Teoria Politica Formal \textcolor{red}{DONE}
\item Depois  passo para Abordagem da Escolha Racional como um subgrupo, e o
  mais importante da teoria formal \textcolor{red}{DONE}
\item Discuto teoria da escolha racional thick and thin \textcolor{red}{DONE}
\item Apresento o que sao preferencias; \textcolor{red}{DONE}
\item Falo em utilidade \textcolor{red}{DONE}
\item Falo em atualização bayesiana \textcolor{red}{DONE}
\item Começo a falar em teoria espacial: apresentação informal. Cito os
  trabalhos base \textcolor{red}{DONE}
\item Falo em abordagem fracamente espacial e fortemente \textcolor{red}{DONE}
\item Apresento a fraca: alternativas como espaço + agentes tem ponto ideal \textcolor{red}{DONE}
\item Forte: agentes tem função de utilidade espacial \textcolor{red}{DONE}
\item Apresento as funções de utilidade comumente usadas, para ilustrar \textcolor{red}{DONE}
\item Falo em aplicação em comite ou eleições  \textcolor{red}{DONE}
\item Ai que entra a distribuição de preferencias da população \textcolor{red}{DONE}
\item Falo do fato de Downs considerar isso importante \textcolor{red}{DONE}
\item Falo da limitação da literatura em não considerar a complexidade da
  situação $\rightarrow$ no caso, vou levantar o argumento de Scott Page de que incerteza
  não é o bastante. \textcolor{red}{DONE}
\item Mostro a distribuição empirica para os 20 paises, discuto o fato do
  formato ser importante, aqui vou puxar o argumento sobre otimo local.\textcolor{red}{DONE}
\item Discuto que também é de se esperar que ela seja estavel mas não estatica
  $\rightarrow$ mostro a distribuição empirica. \textcolor{red}{DONE}
\item Discuto instabilidade micro e estabilidade macro $\rightarrow$ discuto a relação
  entre preferencias e crenças, puxando de binmore e do debate em filosofia. \textcolor{red}{DONE}
  Aqui introduzo OD\textcolor{red}{DONE}
\item Discuto o foco generativo do trabalho. \textcolor{red}{DONE}
\end{itemize}


\section*{Capítulo 2 - Revisão de OD }

\begin{itemize}
  \item Puxar de De marchi e Page os elementos de um abm \textcolor{red}{DONE}
  \item Tirar a citação a semantica desnecessaria \textcolor{red}{DONE}
  \item {\Large Corrigir logo o que Andre indicou.} \textcolor{red}{DONE}
      \begin{itemize}
      \item coloquei no corpo o que tenho que mexer.
    \end{itemize}
  \item Ajeitar o treco de ising, afinal scx  \textcolor{red}{done0}  
    \begin{itemize}
    \item ver na pasta de scx as referencias 
    \end{itemize}
  \item sacar minhas notas novas sobre OD \textcolor{red}{done}
  \item Discutir modelos canônicos. {\Large Sacar os originais}: \textcolor{red}{DONE}
  \begin{itemize}
  \item tirar o q-voter \textcolor{red}{done}
  \item discutir bem melhor os contínuos \textcolor{red}{done}
  \item tirar os de política, ta so enxição de linguiça \textcolor{red}{done}
  \end{itemize}


\item Discuto pluralismo de approachs \textcolor{red}{done}
\item Discuto que a abordagem canonica é fisicalista \textcolor{red}{done}
\item Discuto o espectro de abordagens \textcolor{red}{done}
\item Discuto Modelos que abrem a caixa preta; \textcolor{red}{done}
\item Discuto o problema da dimensionalidade e o rule of thumb, usando bendor, \textcolor{red}{done}
  ostrom, de marchi, e Zaller. \textcolor{red}{done}
\item Puxo o framework de andre \textcolor{red}{done}
\item Discuto Andre 2009 \textcolor{red}{done}
\item Discuto algo sobre agentes imperfeitamente bayesianos. \textcolor{red}{done}

  
\end{itemize}

\section*{Capítulo 3 - Apresentação dos Modelos}

Modelo + Resultados Parciais (simular as relações + apresentar + discutir)
\textcolor{red}{DONE}



\begin{itemize}
\item Implementar $\rho$ \textcolor{red}{DONE}
\item Separar modulo de modelo de modulo de analise \textcolor{red}{DONE}
  \begin{itemize}
  \item Na real fiz bem mais que isso kkkk
  \end{itemize}

\item Colocar titulo relacionado aos parametros no modulo de analise (usar o
  pacote de parametros via struct) \textcolor{red}{DONE}
  \begin{itemize}
  \item Coloquei no name do arquivo. Isso dos parametros vou ver ainda...
  \end{itemize}

\item Ver como separar rodar a simulação de fazer os gráficos
\item Quais os gráficos que vou fazer? Time series, Histogram, Box-Plot!

\item Generalizar a rede: ver metagraphs.
\item \textcolor{green}{Opcional:} ver como ajeitar a performance.
\item O performance bottleneck no momento é fazer os graficos... 
\end{itemize}




\section*{Considerações Finais}

Limitações, o que falta, cronograma.



\end{document}
%%% Local Variables:
%%% mode: latex
%%% TeX-master: t
%%% End:
