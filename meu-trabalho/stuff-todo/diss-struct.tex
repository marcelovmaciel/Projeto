\documentclass{article}
\usepackage[utf8]{inputenc}

% This allows us to add our colors
\usepackage{xcolor}
\usepackage{tcolorbox}
%just input either solarized light or dark to change between the colors
%\input{solarizeddark.tex}
\input{solarizeddark.tex}
\input{styling.tex}


\title{\textcolor{red}{Estrutura da Dissertação}}
\author{Marcelo Veloso Maciel}
\date{}

\begin{document}

% Global Page color base03
\pagecolor{base03}
% Global Text color base1
\color{base1}
\maketitle

\newtcolorbox{mybox}[1]{colback=base03,
  colframe=base00,fonttitle=\bfseries, title=#1}

\begin{abstract}
  Estrutura da Dissertação a partir da conversa com Fernando
\end{abstract}

\section*{Introdução}

Aqui apresento o tema de maneira generica (opinião em geral, opinião publica,
nexo democratico, distribuição de preferencias) e apresento o objetivo do
trabalho :

\begin{mybox}{Objetivo}
  \textcolor{base1}{\textbf{O trabalho tem por objetivo explorar, por meio de
      simulação, a geração de distribuições de preferências da população.}}
  \begin{mybox}{Objetivos Específicos}
    \textcolor{base1}{\begin{itemize}
    \item Simular, por meio de modelos baseados em agentes, a geração de
      distribuições de preferências, formalizadas como pontos ideais.
    \item Dentro do quadro estabelecido por Andre (2012), e a partir do modelo
      de 2009 simular uns variantes (mudança aleatória + topologias distintas)
    \item Validar esses variantes: ver qual melhor se aproxima da distribuição
      empírica;
    \end{itemize}}
    \vspace{0.2cm}
  \end{mybox}
\end{mybox}

\section*{Capítulo 1 - Distribuição de preferencias e Teoria Espacial }

Discutir teoria espacial e a importancia de estudar distribuições

\section*{Capítulo 2 - Revisão de OD }

Discutir modelos canônicos, discutir a relação entre preferencias e opiniões,
discutir porque o framework de andre.


\section*{Capítulo 3 - Apresentação dos Modelos}

Modelo + Resultados Parciais


\section*{Considerações Finais}

Limitações, o que falta, cronograma.



\end{document}